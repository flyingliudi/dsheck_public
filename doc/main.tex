\documentclass[11pt]{article}

\textwidth 6.5in \topmargin -0.5in \oddsidemargin -0.2in \textheight 9in
\usepackage[round]{natbib}
\usepackage{lmodern}
\usepackage{setspace}
\usepackage{amsfonts}
\usepackage{amssymb}
\usepackage{amsmath}
\usepackage{amsthm}
\usepackage{graphicx}
\usepackage{subfig}
\usepackage{comment}
\usepackage{float}
\usepackage{listings}
\usepackage[bookmarks=true]{hyperref}
\usepackage[round]{natbib}
\usepackage{color}
\usepackage{multirow}
\usepackage{lscape}
\usepackage[ruled,vlined]{algorithm2e}
\usepackage[titletoc]{appendix}
\usepackage{stata}
\usepackage{longtable, ltcaption}



\title{DS-HECK: Double-Lasso Estimation of Heckman Selection Model}


\author{\textbf{Masayuki Hirukawa}\footnote{Faculty of Economics, Ryukoku
University, Kyoto, Japan; email: \textsf{hirukawa@econ.ryukoku.ac.jp}} 
\and
\textbf{Di Liu}\footnote{Stata Corp, College Station, Texas, USA; email:
\textsf{flyingliudi@gmail.com}} 
\and 
\textbf{Irina Murtazashvili}\footnote{Drexel University, LeBow College of
business, Philadelphia, USA; email: 
\textsf{im99@drexel.edu}}
\and
\textbf{Artem Prokhorov}\footnote{University of Sydney Business School,
Australia; CEBA; CIREQ; email: \textsf{artem.prokhorov@sydney.edu.au} } }

\date{\firstdraft \quad \thisdraft}


\newcommand{\En}{ \mathbb{E}_N }
\newcommand{\Ind}{\mathbb{I}}
\newcommand{\E}{ \mathop{\mathbb{E}} }
\DeclareMathOperator*{\argmin}{arg\,min}
\newtheorem{prop}{Proposition}
\newtheorem{defi}{Definition}
\newtheorem{assumption}{Assumption}
\newcommand{\dvar}{\mathbf{d}}
\newcommand{\zvar}{\mathbf{z}}
\newcommand{\uvar}{\mathbf{u}}
\newcommand{\alphab}{\mathbf{\alpha}}
\newcommand{\betab}{\mathbf{\beta}}
\newcommand{\etab}{\mathbf{\eta}}
\newcommand{\thetab}{\mathbb{\theta}}
\newcommand{\dsheck}{{\bf DS-HECK}}
\newcommand{\kd}{k_d}
\newcommand{\kz}{k_z}
\newcommand{\xvar}{\mathbf{x}}
\newcommand{\yvar}{\mathbf{y}}

\newcommand{\dil}[1]{{\color{red}#1}}

\DeclareUnicodeCharacter{2212}{-}


%\newcommand*{\thisdraft}{This draft: October 3, 2021} 
%\newcommand*{\firstdraft}{June 2022}  
\date{July 2022}
\begin{document}
\maketitle

\begin{abstract} 
We extend the Heckman (1979) sample selection model by allowing for a large number of controls  that are selected using lasso under a sparsity scenario. %the assumption that the only a few of these controls are important. 
The standard lasso estimation is known to under-select causing an omitted variable bias in addition to the sample selection bias. We outline the required adjustments needed to restore consistency of lasso-based estimation and inference for vector-valued parameters of interest in such models. The adjustments include double lasso for both the selection equation and main equation and a correction of the variance matrix. We also connect the estimator with results on redundancy of moment conditions. We demonstrate the effect of the adjustments using simulations and we investigate the determinants of female labor market participation and earnings in the US using the new approach. The paper comes with {\tt dsheckman}, a dedicated Stata command for estimating double-selection Heckman models. 
\medskip

\noindent JEL Codes: C13

\noindent Key Words: Heckman, probit, double lasso, post selection inference
		
\end{abstract}

\newpage \setstretch{1.5}
\normalsize

%------------------------------------------------------------------------------
\section{Introduction}
%------------------------------------------------------------------------------

In this paper we consider an extension of the familiar \cite{Heckman1979} sample selection model, %. \textbf{DELETE: In its traditional forms, it} \textbf{ADD: which} 
which can be written as follows
\begin{eqnarray}
	y_{1i} &=& \xvar_{1i}'\alphab + u_{1i},  
	\label{eq:y1}\\
y_{2i} &= &\Ind(\xvar_i'\betab+  \zvar_i' \etab + u_{2i} \geq 0),
	\label{eq:y2}
\end{eqnarray}
where  $\Ind(\cdot)$ in the selection equation (\ref{eq:y2}) denotes the indicator function, which takes the value one if its argument is true and zero otherwise, $(u_{1i}, u_{2i})$ are the error terms and the outcome variable $y_{1i}$ is observed only if the selection variable $y_{2i} = 1$. The main equation (\ref{eq:y1}) contains a %(low-dimensional) 
$k_1 \times 1$ vector of explanatory variables $\xvar_{1i}$ and we are interested in estimating and testing the coefficient vector $\alphab$. The explanatory variables of the selection equation are separated into two parts, $\xvar_i=(\xvar_{1i},\xvar_{2i})$ and $\zvar_i$. In the traditional version of the model there is no distinction between $\xvar_{2i}$ and $\zvar_i$ -- both represent the explanatory variables that are present in the selectivity model but not in the main equation of interest. These are the well known \emph{exclusion restrictions} that facilitate identification of $\alphab$. In our setting, for reasons that will become clear shortly we wish to differentiate between $\xvar_{2i}$ and $\zvar_i$. 

Our extension is to introduce a high-dimensional vector of the explanatory variables in the selectivity model (\ref{eq:y2}) which may or may not belong to the model. The vector $\xvar$ is a
low-dimensional $k \times 1$ vector of selection determinants that we wish to keep in the model no matter what. %of interest such that $\xvar_1$ is a subset of $\xvar$. 
The vector $\zvar$ is a high-dimensional $p \times 1$ vector of potential
controls, where $p$ can be as large as the (pre-selection) sample size $N$ or larger and where we do not know which of these controls are important, if any.  The vector
$\betab$ is a $k \times 1$ vector of coefficients on $\xvar$, which can be a target of inference too. The vector $\etab$, on the contrary, is just a $p \times 1$ nuisance parameter vector. 

This extension has many empirical applications in economics where we have a well defined list of regressors for the main equation which has roots in economic theory (e.g., consumer and labor theory) while what determines selection into the sample is less certain \cite[see, e.g.,][]{roy:51, heckman/honore:90}. The classic examples are the estimation of the female labor supply function and  wage functions \citep[see, e.g.,][]{Heckman1979, arellano/bonhomme:17}, which may be subject to selection bias as determinants of the sample selection are confounded with the behavioral functions of interest.
We return to women's labor force participation and labor supply decisions in our empirical application section.

Our objective is to consistently estimate $\alphab$ in the outcome equation
(\ref{eq:y1}) under a potential sample selection bias arising from the fact that in the observed sample
\[\mathbb{E}(y_{1i}| \xvar_i, \zvar_i, y_{2i}=1) = \xvar_{1i}'\alphab + \mathbb{E}(u_{1i}|  \xvar_i, \zvar_i, y_{2i}=1) \ne \xvar_{1i}'\alphab, \]
unless $\mathbb{E}(u_{1i}|  \xvar_i, \zvar_i, y_{2i}=1)=0$, which is a questionable assumption in practice. \cite{Heckman1979} assumed joint normality of $(u_{1i}, u_{2i})$ and showed that $\mathbb{E}(u_{1i}|  \xvar_i, \zvar_i, y_{2i}=1)=\gamma \lambda(\xvar_i'\betab+  \zvar_i' \etab)$, where $\lambda(\cdot) = \phi(\cdot)/\Phi(\cdot)$ is known as the inverse Mills ratio. The two-step heckit procedure is (a) to run the maximum likelihood estimation (MLE) for the probit of $y_{2i}$ on $(\xvar_{i}, \zvar_i)$ and use the estimates $(\hat{\betab}, \hat{\etab})$ to obtain $\hat{\lambda}_i \equiv \lambda(\xvar_i'\hat{\betab}+  \zvar_i' \hat{\etab})$ and then (b) to regress $y_{1i}$ on $x_{1i}$ and $\hat{\lambda}_i$. Under correct specification, the resulting estimators $\hat{\alphab}$ and $\hat\gamma$ are consistent and the usual t-test on $\hat\gamma$ can be used to test for selection bias. If the null of no bias is rejected, the standard errors of the second step have to be corrected for the first step estimation error which is done via a full MLE using normality of the errors or via an analytic correction to the variance in the second step. 

The high-dimensionality of
$\zvar_i$ poses a challenge in applying the traditional two-step procedure. First, we cannot include all the variables in $\zvar_i$ in the first step because there are too many variables. If $p$ is larger than $N$, the probit with all $\xvar_i$ and $\zvar_i$ is
infeasible, and even if $p$ is substantially smaller than $N$ but is large then including all these variables can cause difficulties in MLE convergence. 

In order to make estimation feasible, it is common to impose a certain structure on $\eta$, known in the literature on regularized estimation as a sparsity scenario.  In particular, we assume that only a few elements in the coefficient vector $\eta$ are substantially different from zero. % or approximately zero. 
Although we assume $\eta$ is sparse, we do not know which elements are non-zero
and a consistent model selection technique is required.  A popular approach to regularizing linear models is the least
absolute shrinkage and selection operator (lasso) developed by
\cite{Tibshirani1996}. The method penalizes the objective function with an $l_1$-norm of the coefficients. This shrinks the irrelevant coefficients to zero and thus serves as a model selection tool. However, even for purely linear models, this approach has well known challenges.  

First, lasso makes mistakes. Failure to account for the fact that the covariates have been selected by lasso results in invalid inference.  The reason is that lasso, like many other model selection techniques, does not always find all the relevant covariates especially when some coefficients are small. Model selection mistakes made by lasso cause the distribution of this naive estimator to be biased and nonnormal. For example,  \cite*{LPotscher2008b},
\cite*{LPotscher2008}, and \cite*{PLeeb2009} %\textbf{DELETE: showed and ADD: show} 
show that the normal approximation for the 
naive lasso estimator will produce misleading inference.
\cite*{bellonichernozhukovhansen2014}, \cite*{bellonichernozhukovwei2016}, and
\cite{cherno/etal:18} derive estimators that are robust to the mistakes made by lasso.  Such robust estimators are often referred to as Neyman orthogonal (NO) estimators because they can be viewed as extensions of an approach proposed by \cite*{Neyman1959}.

The second challenge is choosing the lasso tuning parameter.  Lasso's ability to select relevant covariates depends on the method used to choose the tuning parameters.
\cite*{bellonichernozhukovhansen2014} propose a plug-in method and show that NO estimators perform well on linear models under that method.  \cite*{bellonichernozhukovwei2016} extend the linear lasso to logit models and show good performance using a simplified version of the plug-in method.  \cite*{drukker/liu:22} extend the plug-in method to cross-sectional generalized linear models and  provide Monte Carlo evidence that their extension works well in finite samples.  


In this paper, we develop NO estimation for the model in (\ref{eq:y1})-(\ref{eq:y2}) which we call double-selection Heckman procesure, or DS-HECK. % for the coefficients of interest, which is $\alphab$ in the outcome Equation (\ref{eq:y1}). 
The DS-HECK %{\dsheck} 
estimator draws upon the
classical two-step heckit estimator and the
double-selection lasso for the high-dimensional generalized linear
models proposed by \cite*{bellonichernozhukovwei2016}. We detail the steps involved in the estimation, work out the estimator properties and derive the variance corrections. We also provide new insights into how NO estimation is linked to results on redundancy of knowledge in moment-based estimation considered by \cite*{breusch/etal:99} and \cite{prokhorov/schmidt:09}.


The rest of the paper is organized as follows. Section \ref{sec:model} describes and studies the DS-HECK %{\dsheck} 
estimator.  In Section \ref{sec:sim}, we present simulation results that demonstrate an excellent performance of DS-HECK %{\dsheck} 
in finite samples. In
Section \ref{sec:app}, we apply DS-HECK %{\dsheck} 
to estimate married women's wage using the 2013 PSID wave, 
in the presence of high-dimensional controls and potential sample selection bias. Finally, Section \ref{sec:conclude} concludes. 

		
%------------------------------------------------------------------------------
\section{The {\dsheck} estimator}\label{sec:model}
%------------------------------------------------------------------------------

%------------------------------------------------------------------------------
	\subsection{Settings}
%------------------------------------------------------------------------------

%Formally, the high-dimensional Heckman sample selection model is 
%\begin{align*}
%	y_{1i} &= \xvar_{1i}'\alphab + u_{1i} \\ 
%	y_{2i} &= \Ind(\xvar_i'\betab+  \zvar_i' \etab + u_{2i} \geq 0) 
%\end{align*}
We maintain the standard assumption of the Heckman sample selection
model.

\begin{assumption} \label{assump:heckman}
(a) $(\xvar, \zvar, y_2)$ are always observed, $y_1$ is observed only
when $y_2 =1$;
(b) $(u_1, u_2)$ is independent of $\xvar$ and $\zvar$ with zero mean;
(c) $u_2 \sim N(0, 1)$;
(d) $\E(u_1 | u_2) = \gamma_1 u_2$.
\end{assumption}

Assumption \ref{assump:heckman} is in essence the same as in
\citet[][p.~803]{wooldridge2010}. Part (a) describes the nature of sample
selection. Part (b) assumes that $\xvar$ and $\zvar$ are exogenous. Part
(c) is restrictive but needed to derive the conditional expectation of
$y_1$ given that it is observed. Part (d) requires linearity in the conditional
expectation of $u_1$ given $u_2$, and it holds when $(u_1, u_2)$ is bivariate normal. However, it also holds under weaker assumptions
when $u_1$ is not normally distributed.

%In order to consistently estimate $\alpha$, we will see that we need to
%estimate $\betab$ consistently. 
Additionally, we impose a sparsity scenario on $\etab$. %, as in \cite{belloni2016}, we assume it is sparse in the following sense.

\begin{assumption} \label{assump:sparsity}
$\eta$ is sparse; that is, most of the elements of $\etab$ are zeros. Namely, $||\etab||_0 \leq
s$, where $||\cdot||_0$ denotes the number of non-zero components of a vector. We require $s$ to be small relative to the sample size $N$. In particular, 
$\frac{s^2\log^2(max(p , N))}{N} \longrightarrow 0$.
\end{assumption}

This assumption follows \cite{bellonichernozhukovwei2016}. In the settings of generalized linear models, it allows for the estimation of the nuisance parameter in the selection equation at the rate
$o(N^{-1/4})$ (see their Condition IR). In our settings, this rate is needed to guarantee the consistent estimation
of $\betab$. % as stated in the Condition IR in \cite{bellonichernozhukovwei2016}.

Under Assumption \ref{assump:heckman}, it is easy to show that %the outcome Equation (\ref{eq:y1}), and
%the selection Equation (\ref{eq:y2}), we can write the the conditional
%expectation of $\yvar_1$ given $\xvar$, $\zvar$, and $\yvar_2 = 1$ as
\begin{align}
\E(y_{1} | \xvar, \zvar, y_2 = 1) = \xvar_1'\alpha 
	+ \gamma \lambda(\xvar'\betab + \zvar'\eta), \label{eq:y1cond}
\end{align}
where $\lambda(\cdot) = \phi(\cdot)/\Phi(\cdot)$ is the inverse Mills ratio.
%For the derivation of Equation (\ref{eq:y1cond}), see \cite{Heckman1979} or
%\cite{wooldridge2010}.
In essence, this is the classic formulation of \cite{Heckman1979} where the presence of $\xvar_2$ and $\zvar$ in the selection equation (but not in the main equation of interest) permits estimation of the model even when the inverse Mills ratio is close to being linear in its argument\cite{}. 


We wish to explore the behavior of this conditional expectation with respect to potential errors in the choice of $\zvar$. It is easy to see from applying the mean-value theorem to the inverse Mills ratio evaluated at  % study the Suppose now we know we know the true value of $\betab$, and we only use 
%
$\xvar'\betab$, % to compute the inverse mills ratio, what would the Equation\
that we can rewrite (\ref{eq:y1cond}) as follows % become? We can use the taylor expansion to expand $\lambda(\xvar'\betab + \zvar'\eta)$ at the point $\xvar'\betab$.
\begin{align}
\E(y_{1} | \xvar, \zvar, y_2 = 1) 
%	&= \xvar_1'\alphab + \gamma \lambda(\xvar'\betab + \zvar'\eta) \nonumber	\\
	&= \xvar_1'\alphab + \gamma \lambda(\xvar'\betab) 
	+\gamma \lambda^{(1)}(q) \zvar'\eta \nonumber \\
	&= \xvar_1'\alphab + \gamma \lambda(\xvar'\betab) 
	+ \zvar'\omega  \label{eq:y1cond2}
\end{align}
where $q$ is a point between $\xvar'\betab + \zvar'\eta$ and $\xvar'\betab$,
$\lambda^{(1)}(\cdot)$ is the first-order derivative of $\lambda(\cdot)$,  and
$\omega = \gamma \lambda^{(1)}(q) \eta$. It is well known that $\lambda^{(1)}(\cdot)$ is monotone and bounded between -1 and 0 \cite[see, e.g.,][]{sampford:53}. We note that $\omega$ %is a randomvcoefficient vector which 
depends on $q$,  $\gamma$ and $\eta$, and that  %From the form of $\omega$, it follows that 
if $\eta$ sparse than $\omega$ is sparse too. 
\begin{prop} \label{prop:sparseomega}
The vector $\omega$ in Eq.~(\ref{eq:y1cond2}) inherits the same  sparsity properties as the vector $\eta$, i.e.~$||\omega||_0=||\etab||_0$.
\end{prop}

\textbf{Proof.} Sketches of the proofs of all less obvious propositions are given in the Appendix. 

This Proposition makes it clear that a Heckman model with sparsity in the selection equation can be written as a heckit model with the same sparsity scenario in the main equation of interest. %Following \cite{Heckman1979}, we note that the near linearity of implied near constancy of This justifies the use of lasso on a modified version of the standard heck-it model in which we remove the selectivity controls $\mathbf{z}$ from the Heckman correction and add them as controls to the main equation. 
 %the alternative approaches to estimating this model. %the effects of this modification on consistency of the estimation. 
Next we derive some conditions on the linear approximation of the inverse Mills ratio using $\zvar$ in the selected sample which is common for lasso-based model selection but new in the context of the Heckman model.

Let $n$ denote the size of the selected sample, defined as follows, 
\[n = \sum_{i=1}^N\mathbb{I}(y_{2i}=1).\]
Then, for the selected observations, we can write
\[y_{1i} = \xvar_{1i}'\alpha + g(\xvar_i, \zvar_i) + \epsilon_i,\]
where $g(\xvar_i, \zvar_i)=\gamma \lambda(\xvar_i'\beta+ \zvar_i'\eta)$, $\mathbb{E}(\epsilon_i|\xvar_i, \zvar_i) = 0$ and $\mathbb{V}(\epsilon_i|\xvar_i, \zvar_i) = \sigma^2$. 
We follow \cite{bellonichernozhukovhansen2014} and write $g(\xvar_i, \zvar_i)$ in a linear form subject to a bound on the approximation error:
\begin{equation}\label{eq:r}y_{1i} = \xvar_{1i}'\alpha + \gamma \lambda(\xvar_i'\beta) + \zvar_i'\delta + r_i + \epsilon_i,
\end{equation}
where $r_i, i=1,\ldots, n,$ is the approximation error such that $\sqrt{\frac{1}{n}\sum_{i=1}^n \mathbb{E}r_i^2} = O\left(\sqrt{\frac{s}{n}}\right)$. Additionally, we assume that the selected and pre-selection sample sizes are of the same order.
\begin{assumption} \label{assump:r&n}
Eq.~(\ref{eq:r}) holds in the selected sample with $n=O(N)$ and 
\[\sqrt{\frac{1}{n}\sum_{i=1}^n \mathbb{E}r_i^2} = O\left(\sqrt{\frac{s}{n}}\right).\]
\end{assumption}
The assumption on the approximation error follows \cite{bellonichernozhukovhansen2014}. Similar to Assumption \ref{assump:sparsity}, Assumption \ref{assump:r&n} ensures that we can estimate the nuisance parameter in the selected sample at the rate $o(n^{−1/4})$ rate. In the context of the Heckman model, it implies that $||\delta||_0=||\omega||_0=||\eta||_0 \le s$ and that $\delta$ is also estimated at $o(N^{-1/4})$ since  $n = O(N )$.

Next we investigate how to consistently estimate this model accounting for the high-dimensional nuisance parameter in both equations.

\begin{comment}

\begin{prop} \label{prop:sampford}
The vector $\omega$ in Eq.~(\ref{eq:y1cond2}) inherits the same  sparsity properties as the vector $\eta$, i.e.~$||\omega||_0=||\etab||_0$.
\end{prop}

In
order to apply the double-selection Lasso in \cite{belloni/etal:13} to a linear
model, we assume that $\zvar'\omega$ can be approximated by a sparse linear
model.

\begin{assumption} \label{assump:sparsity2}
We assume $\zvar'\omega = \zvar'\zeta + \epsilon_z$, with $\zeta$ to be a sparse
non-random coefficient vector, and $\epsilon_z$ is mean-zero approximation
errors uncorrelated to $\zvar$ and $\xvar$. 
\end{assumption}

\dil{(!! We need to justify Assumption \ref{assump:sparsity2}.  Suppose $\zeta =
\gamma \E(\lambda^{(1)}(q)) \eta$, then it is indeed sparse if $\eta$ is sparse.
However, it implies $\epsilon_z = \zvar \gamma (\lambda^{(1)}(q) -
\E(\lambda^{(1)}(q))) \eta$, and $\epsilon_z$ will not be independent from
$\xvar$ and $\zvar$ because the point $q$ depends on them.)}
Cite the boundedness result from Sampford 1953.

Under Assumption \ref{assump:sparsity2} and Equation (\ref{eq:y1cond}), we can
write $\E(\yvar_{1} | \xvar, \zvar, y_2 = 1)$ as
\begin{align}
\E(\yvar_{1} | \xvar, \zvar, y_2 = 1) 
	&= \xvar_1'\alphab + \gamma \lambda(\xvar'\betab) 
	+ \zvar'\zeta  \label{eq:y1cond2linear}
\end{align}

Under Assumption \ref{assump:heckman} and Equation (\ref{eq:y2}), we can write 
$\Pr(\yvar_2 = 1 | \xvar, \zvar)$ as
\begin{align}
\Pr(\yvar_2 = 1 | \xvar, \zvar) = \Phi(\xvar'\beta + \zvar'\eta)
\label{eq:y2cond}
\end{align}

\end{comment}
%------------------------------------------------------------------------------
	\subsection{Estimation of the selection equation}
%------------------------------------------------------------------------------


%Equations (\ref{eq:y1cond2}) and (\ref{eq:y2cond}) are our key intuition to
%design the proposed {\dsheck} estimator.  Here is the logic.

Clearly if we knew the true value of $\betab$, we could treat $\lambda(\xvar'\betab)$
as a known variable and we could estimate $\alphab$ and $\gamma$, treating $\delta$ as nuisance parameters. So we start with a consistent estimation of $\betab$ in Eq.~(\ref{eq:y2}) using the approach of \cite{bellonichernozhukovwei2016} combined with parameter tuning of \cite{drukker/liu:22}.


The estimation involves three steps:

						%% algo dsprobit
%\begin{algorithm}[H]\label{algo:dsprobit}
%\caption{Double selection probit
%(based on the double-selection logit in \cite{belloni2016})}
	\begin{description}
%%%
	\item{\bf Step 1 (post-lasso probit)} We start by estimating a penalized probit of $y_2$ on $\xvar$ and $\zvar$ using the lasso penalty:
	\begin{align*}
		(\hat{\betab}, \hat{\etab}) &= 
		\argmin_{\betab, \etab} 
		\En(\Lambda_i(\betab, \etab)) + \lambda_1 ||(\betab,
		\etab)||_1, 
	\end{align*}
	where $\En$ denotes the sample mean of $N$ observations, $\Lambda_i(\cdot)$ is the negative log-likelihood for the probit
	model, $||\cdot||_1 $ is the lasso ($l_1$) norm of the parameters and $\lambda_1$ is a tuning parameter chosen using the plug-in method of 
	\cite{drukker/liu:22}. This produces a subset of the variables in $\zvar$ indexed by $support(\hat{\etab})$, where for a $p$-vector $v$, $support(v):=\{ j \in \{1, ..., p\}: v_j \ne 0\}$. These variables are used in the post-lasso probit:
	\begin{align*}
		(\tilde{\betab}, \tilde{\etab}) &= 
		\argmin_{\betab, \etab} 
		\En(\Lambda_i(\betab, \etab)): support(\etab) \subseteq
		support(\hat{\etab}) 
	\end{align*}
    As a result, we obtain the sparse probit estimates $(\tilde{\betab}, \tilde{\etab})$ where $\tilde{\etab}$ contains only a few non-zero elements. \cite{bellonichernozhukovwei2016} propose using these estimates to construct weights
	$\hat{f}_i = \hat{w}_i/\hat{\sigma}_i$, where $\hat{w}_i =
	\phi(\xvar_i'\tilde{\betab} + \zvar_i'\tilde{\etab})$, and 
	$\hat{\sigma}_i^2 = \Phi(\xvar_i'\tilde{\betab} + \zvar_i'\tilde{\etab})
	(1 - \Phi(\xvar_i'\tilde{\betab} + \zvar_i'\tilde{\etab}))$, for $i=1,\ldots, N$.

%%%
	\item{\bf Step 2}. We use the weights from Step 1, to run a weighted lasso regression in which for each variable $x_j$ in $\xvar$, $j=1, \ldots, k$, we run the penalized regression of $\hat{f}_i x_{ij}$ on $\hat{f}_i \zvar_i$,
	\begin{align*}
	\hat{\theta}_j = \argmin_{\theta_j} 
	\En (\hat{f}_i^2(x_{ij} - \zvar_i'\theta_j)^2) 
	+ \lambda_2||\theta_j||_1,
	\end{align*}
	where $\lambda_2$ is chosen by the plug-in method of \cite{drukker/liu:22}. For each element of $\xvar$, this produces a selection from the variables in $\zvar$ indexed by $support({\hat{\theta}}_j), j=1, \ldots, k$.   

%%%
	\item{\bf Step 3 (double-selection probit)}. We use the variables selected from $\zvar$ in Steps 1 and 2 to run the probit of $y_2$ on $\xvar$ and the union of the sets of variables selected in Steps 1 and 2:
	\begin{align*}
	(\check{\betab}, \check{\eta}) 
	= \argmin_{\beta, \eta} \En (\Lambda_i(\betab, \eta)\hat{f}_i/\hat{\sigma}_i), 
	\end{align*}
	where $support(\eta) \subseteq 
	\left(support({\hat{\etab}})
	\cup support({\hat{\theta}}_1) 
	 \cup \ldots \cup support(\hat{\theta}_J)\right)$.
	\end{description}
%\end{algorithm}

In the setting of generalized linear models, \cite{bellonichernozhukovwei2016} show that the double-selection probit corrects for the omitted variable bias introduced by a naive application of lasso to Eq.~(\ref{eq:y2}). The intuition is that the double selection using weights $\hat{f}_i$ ``neymanizes'' Step 3. That is, it ensures that the estimation error from the first step does not affect the estimated parameter vector of the last step. 

It follows from \citet[][Theorem 1]{bellonichernozhukovwei2016} that, under Assumption \ref{assump:sparsity}, 
$\check{\betab}$ is a consistent estimator of  $\betab$ and its variance can be obtained from Step 3 using the well known ``sandwich'' formula for probit. For example, in Stata it can be obtained using the \texttt{vce(robust)} syntax. We obtain a regular $\sqrt{N}$-consistent estimator of $\beta$ with standard inference, even though a penalized non-$\sqrt{N}$ estimator is being used to carry out model selection for the high-dimensional nuisance parameter $\eta$.
% following the \texttt{probit} command. 
%We
%can apply the double-selection lassos to Equation (\ref{eq:y1cond2}) for the
%inference for $\alphab$ and $\gamma$ in the presence of high-dimensional and
%sparse $\zeta$; see \cite*{bellonichernozhukovhansen2014}.

%However, we do not know the true value of $\betab$. Instead, if we can obtain
%consistent estimate for $\betab$ and its standard errors, we can still apply the
%double-selection lasso method but with variance adjustment. 

%So, the question is reduced to how to obtain valid inference for $\beta$. The
%solution is to apply the double selection Probit to Equation (\ref{eq:y2cond}).
%Double selection Probit is just a particular case of Double selection for
%the generalized linear model proposed in \cite{belloni2016}.


%------------------------------------------------------------------------------
	\subsection{Connection to redundancy of moment conditions}
%--------------------------------------------

\cite{bellonichernozhukovwei2016} use the Neyman orthogonalization to obtain their result. In this section we show how NO argument relates to the concept of moment redundancy pioneered by \cite{breusch/etal:99}. This offers an alternative way of arriving at the weights derived by 
\cite{bellonichernozhukovwei2016}. 

The key insight of \cite{bellonichernozhukovwei2016} is that the weights $f_i$ ensure the validity of the main moment condition: 
\begin{equation}\label{eq:no}
    \mathbb{E} g_i(\beta_0, \eta_0, \theta_0) \equiv \mathbb{E} [y_{2i} - \Phi(\xvar'_i\beta_0 +\zvar_i'\eta_0)](f_i\xvar_i - f_i\zvar_i'\theta_0) = 0, 
\end{equation}
where the instruments $(f_i\xvar_i - f_i\zvar_i'\theta_0)$ are chosen so that   
\[\mathbb{E}\frac{\partial}{\partial \eta} g_i(\beta_0, \eta_0, \theta_0) = 0.\]
It is easy to see that for $f_i = \frac{\phi(\xvar'_i\beta_0 +\zvar_i'\eta_0)}{\sigma_i^2}$, where $\sigma_i^2=\mathbb{V}(y_i|\xvar_i, \zvar_i)=\Phi(\xvar_i'{\beta}_0 + \zvar_i'{\eta}_0)
	(1 - \Phi(\xvar_i'{\beta}_0 + \zvar_i'{\eta}_0))$, the zero expected derivative condition holds. See Eq.(28) of \cite{bellonichernozhukovwei2016}.

What has not been noted is that these conditions correspond to what \cite{prokhorov/schmidt:09} call \emph{moment and parameter redundancy} (M/P-redundancy), that is the situation when neither the knowledge of the additional moment conditions nor the knowledge of the parameter they identify help improve efficiency of estimation. 


To see this, let $\xvar_i$ be a scalar for notational simplicity, and write the moment conditions identifying $\beta$ and $\eta$ as follows
%\begin{equation}\label{eq:h1}
%    \mathbb{E}h_{1i}(\theta_0) \equiv  \mathbb{E} f_i \zvar_i(f_i\xvar_i - f_i\zvar_i'\theta_0) = 0
%\end{equation}
%denote the moment conditions that identify $\theta$. This is the moment condition used to generate the regressor for the second step of a two-step estimation. 
%The second step uses the moment condition 
\begin{eqnarray}\label{eq:gmm1}
\mathbb{E}h_{1i}(\beta_0, \eta_0) \equiv \mathbb{E}(y_{2i}-\Phi(\xvar_i' \beta_0 + \zvar_i'\eta_0) )f_i\xvar_i = 0,\\\label{eq:gmm2}
\mathbb{E}h_{2i}(\beta_0, \eta_0) \equiv \mathbb{E}(y_{2i}-\Phi(\xvar_i' \beta_0 + \zvar_i'\eta_0) )f_i\zvar_i = 0,
\end{eqnarray}
%to identify $\beta$. 
where the subscript ``0'' on a parameter denotes the true value. 
These moment conditions correspond to the first order conditions of the probit and stem from the specification $\mathbb{P}(y_{2i}=1|\xvar_i, \zvar_i) = \Phi(\xvar'_i\beta_0 +\zvar_i'\eta_0)$. % and it is a function of $\theta$ because $\xvar_i = \zvar_i'\theta_0 + error$. 

This is the system of moment conditions  considered by \cite{breusch/etal:99} %\cite{prokhorov/schmidt:09} 
in the Generalized Method of Moments (GMM) framework. See their Eq.(6). %(3A)-(3B).
They show that the (optimal) GMM estimation based on Eqs.(\ref{eq:gmm1})-(\ref{eq:gmm2}) is equivalent to the estimation based on Eq.(\ref{eq:gmm2}) and the error in the linear projection of Eq.(\ref{eq:gmm1}) on Eq.(\ref{eq:gmm2}). Using their notation, we can write the equivalent moment system as follows
\begin{eqnarray}\label{eq:lp}
\mathbb{E}[h_{1i}(\beta_0, \eta_0)- \Omega_{12}\Omega_{22}^{-1}h_{2i}(\beta_0, \eta_0)] = 0,\\\label{eq:h2}
\mathbb{E}h_{2i}(\beta_0, \eta_0)  = 0,
\end{eqnarray}
where $\Omega_{12}$ and $\Omega_{22}$ are the relevant parts of the moment variance matrix
\[\Omega\equiv \mathbb{V}\left[\begin{array}{c}h_{1i}(\beta_0, \eta_0)\\h_{2i}(\beta_0, \eta_0)\end{array}\right]=\left[\begin{array}{cc}\Omega_{11}&\Omega_{12}\\ \Omega_{21}&\Omega_{22}\end{array}\right].\] 

It is easy to see that Eq.(\ref{eq:lp}) coincides with Eq.(\ref{eq:no}) subject to the additional notation that $\theta'_0 = \Omega_{12}\Omega_{22}^{-1}=\mathbb{E}\sigma^2_i f^2_i \xvar_i \zvar_i' [\mathbb{E}\sigma^2_i f^2_i \zvar_i \zvar'_i]^{-1}$. It is also easy to see that the entire estimation problem can be written in the GMM framework as follows:
\begin{eqnarray}\label{eq:1}
    \mathbb{E} g_i(\beta_0, \eta_0, \theta_0) &\equiv &\mathbb{E} [y_{2i} - \Phi(\xvar'_i\beta_0 +\zvar_i'\eta_0)](f_i\xvar_i - f_i\zvar_i'\theta_0) = 0, \\\label{eq:2}
\mathbb{E}h_{2i}(\beta_0, \eta_0) &\equiv& \mathbb{E}(y_{2i}-\Phi(\xvar_i' \beta_0 + \zvar_i'\eta_0) )f_i\zvar_i = 0\\\label{eq:3}
\mathbb{E}h_{3i}(\theta_0) &\equiv& \mathbb{E}\sigma_i f_i \zvar_i(\sigma_i f_i\xvar_i - \sigma_i f_i\zvar_i'\theta_0)=0,  
\end{eqnarray}
where the first equation is Eq.(\ref{eq:no}) above, % of \cite{bellonichernozhukovwei2016}, 
the second equation is the moment condition that identifes $\eta_0$ and the third equation is a modified (through the inclusion of the scalar $\sigma_i$) version of the OLS first-order conditions used to estimate $\theta_0$. 

We note that, due to a separability result of  \cite{ahn/schmidt:95}, we cannot improve on the estimation of $\theta_0$ by estimating it jointly with $(\beta_0, \eta_0)$ %ased on the same number of moment conditions as the number of elements in $\theta_0$ does not affect the asymptotic distribution of the GMM estimator of $\beta_0$ and $\eta_0$ 
because the additional conditions (\ref{eq:1})-(\ref{eq:2}) only determine $\theta_0$ in terms of 
$\beta_0$ and $\eta_0$. See also \citet[][Statement 6]{prokhorov/schmidt:09}. We further note that by Statement 7 of \cite{prokhorov/schmidt:09}, joint estimation of the entire vector $(\beta_0, \eta_0, \theta_0)$ is equivalent to a two-step estimation where $\theta_0$ is estimated first and the second step is adjusted for the estimation error of the first step. 

More importantly, because the correlation between the moment functions in Eqs.(\ref{eq:1}) and (\ref{eq:2}) is zero and the expected derivative of Eq.(\ref{eq:1}) with respect to $\eta$ is zero, the condition of partial redundancy of \citet[][Theorem 7]{breusch/etal:99} holds (in their notation $G_{21}=0$ and $\Omega_{21}=0$). This means the moment condition (\ref{eq:2}) is redundant for the estimation of $\beta$ (\emph{M-redundancy}). Additionally, these conditions are sufficient to show that the knowledge of the value of $\eta_0$ is redundant (\emph{P-redundancy}). See Statement 4 of \cite{prokhorov/schmidt:09}. So the NO condition of   \cite{bellonichernozhukovwei2016} corresponds to a well established situation in GMM estimation when neither the knowledge of the parameter $\eta_0$, nor the knowledge of the moment condition (\ref{eq:2}) helps estimate $\beta_0$. %l;;lboth the moment conditions  can help estimate en conclude this section by showing that hey sho both the knowledge of $\theta_0$ and the knowledge of Eq.(\ref{eq:h1}) do not help estimate $(\beta, \eta)$ more precisely if $h_{1i}(\theta_0)$ and $h_{2i}(\beta_0, \eta_0, \theta_0)$ are uncorrelated and the expected derivative of $h_{2i}(\beta, \eta, \theta)$

%------------------------------------------------------------------------------
	\subsection{Choice of penalty parameter}
%------------------------------------------------------------------------------
The penalty parameters, $\lambda_1$ and $\lambda_2$ can be derived analytically but typically, data-driven methods are used. Their theoretical validity and practical performance have been well studied. For example, cross-validation or AIC typically under-penalize (over-select) by including too many variables to reduce bias, compared with BIC or plug-in methods. %There are also %Another approach suggested in %Section 4.2 of [13] relies on new Gaussian approximation results and can be implemented via a %multiplier bootstrap procedure. 
Additionally, when too many variables are selected, this violates the sparsity assumption required for the double lasso to work. 

Plug-in methods have been shown to perform well in a multitude of settings %\cite*{drukker/liu:21}, 
\cite[see, e.g.,][]{drukker/liu:22, belloni/cherno/hans/kozbur:16, bellonichernozhukovhansen2014}. 

%------------------------------------------------------------------------------
	\subsection{Estimation of the main equation}
%------------------------------------------------------------------------------


We can now return to the estimation of $\alphab$ and $\gamma$. Similar to \cite{bellonichernozhukovwei2016}, \cite{bellonichernozhukovhansen2014} %\textbf{DELETE: observed and ADD: observe}
observe that the direct application of lasso to linear models with a large-dimensional nuisance parameter results in a biased estimation of the parameter of interest, which in their case is a scalar treatment effect. They propose a double selection procedure. We follow their approach subject to a few modifications that reflect the specifics of our main equation. 

First, with a consistent estimator of $\betab$, a natural estimator of the inverse Mills ratio in Eq.~(\ref{eq:y1cond2}) is as follows: \[\widehat{\lambda(\xvar_i'\betab)}
	= \phi(\xvar_i'\check{\betab})/\Phi(\xvar_i'\check{\betab}).\] 
It is also natural to account for the fact that this is a generated regressor when constructing the variance matrix, something we consider later.

Second, \cite{bellonichernozhukovhansen2014} derive their results for a scalar parameter. Because the variables of interest $\xvar$ %$and $\lambda(\xvar'\betab)$ 
form a vector, we need to extend the
original double selection lasso estimation to vectors. We provide the details of this extension using the NO arguments in Appendix
\ref{sec:dsvec}. 

We can now discuss the estimation of the main equation which combines the double selection lasso of \cite{bellonichernozhukovhansen2014} and parameter tuning by \cite{drukker/liu:22}. It proceeds in three steps: 
\begin{description}

	\item{\bf Step 1}. We run the lasso regression of $\yvar_1$ on $\zvar$
	\begin{align*}
		\check{\theta}_y = \argmin_{\theta_y} \mathbb{E}_n\left[(y_{1i} - \zvar_i'\theta_y)^2\right] 
		+ \lambda_1 ||\thetab_y||_1.
	\end{align*}
	%where $\lambda_1$ is chosen by the plugin method described in
	%\cite{Drukker/liu:22}.
%
This produces a subset of $\zvar$ indexed by $support(\check{\theta}_y)$.
	\item{\bf Step 2}.
	For each variable $x_{1j}$ in $\xvar_1$, $j=1, \ldots, k_1$, we run the lasso regression of $x_{1j}$ on $\zvar$:
	\begin{align*}
		\check{\theta}_j = \argmin_{\theta_j} \mathbb{E}_n\left[(x_{1ij} - \zvar_i'\theta_j)^2\right] 
		+ \lambda_2 ||\theta_j||_1.
	\end{align*}
	Additionally, we run the lasso regression of $\widehat{\lambda(\xvar_i'\betab)}$ on $\zvar$:
	\begin{align*}
		\check{\theta}_j = \argmin_{\theta_\lambda} \mathbb{E}_n\left[(\widehat{\lambda(\xvar_i'\betab)} - \zvar_i'\theta_j)^2\right] 
		+ \lambda_2 ||\theta_\lambda||_1.
	\end{align*}	
This step produces subsets of $\zvar$ indexed by $support(\check{\theta}_j), j=1, \ldots, k_1$, and  $support(\check{\theta}_\lambda)$.	
%
	\item{\bf Step 3}. We run the regression of $y_{1i}$ on $\xvar_{1i}$, $\widehat{\lambda(\xvar_i'\betab)}$, and the union of the sets selected in Steps 1 and 2:
	\begin{align*}
	(\widehat{\alpha}, \widehat{\gamma}, \widehat{\theta})
	=
	\argmin_{\alpha, \gamma, \theta} \mathbb{E}_n \left[(y_{1i} - \xvar_{1i}'\alpha - 
		\widehat{\lambda(\xvar_i'{\betab})}\gamma 
		- {\zvar}_i'\theta)^2\right],
	\end{align*}
	where $support(\theta) \subseteq support(\check{\theta}_y) \cup 
	support(\check{\theta}_1)
	\cup
	\ldots \cup
	support(\check{\theta}_J)
	$

\end{description}


\begin{prop}\label{prop:heckit}
Under Assumptions \ref{assump:heckman}-\ref{assump:r&n}, the %double selection heck-it 
DS-HECK estimation in Steps 1-3 above is consistent for $\alpha$ and $\gamma$ and 
%$(\widehat{\alpha}, \widehat{\gamma}, \widehat{\theta})$ are consistent estimators of 
post-double-selection inference on $\alpha$ and $\gamma$ is valid.
%  but wee need to adjust the variacen due to the estimation of beta.
\end{prop}

The DS-HECK estimator corrects the bias generated by applying the lasso directly to Eq.~(\ref{eq:y1cond2}). The simulation experiments we report in Section \ref{sec:sim} illustrate the size of bias.

Following \cite{bellonichernozhukovhansen2014}, we can claim that inference about the vector $(\alpha', \gamma)$ is valid but, unlike \cite{bellonichernozhukovhansen2014}, it is valid up to the variance matrix correction reflecting the post-lasso probit estimation of $\betab$. 

\begin{comment}

\vskip 1cm

%%%%%%%%%%%%%%%%%%%%%%%%%%%%%%%%%%%%%%%%%%%%%%%%%%%%%%%%%%%%%%%%%%%%%%%%%%%%%%%
	\begin{algorithm}[H]\label{algo:dsheckman}
%%%%%%%%%%%%%%%%%%%%%%%%%%%%%%%%%%%%%%%%%%%%%%%%%%%%%%%%%%%%%%%%%%%%%%%%%%%%%%%
\caption{The {\dsheck} estimator}
\begin{enumerate}
	\item {\bf Choose the proxy variables}.  Use Lasso to set $\xvar$ as
	variables to predict the inverse mills ratio. Notice that we can impose
	$\xvar$ to always include $\xvar_1$ when use lasso to select variables. 

	\item \label{algo:dsheck_step2} {\bf Obtain consistent estimates
	$\check{\betab}$ for $\betab$}.  For details, see the double selection
	probit in Algorithm \ref{algo:dsprobit}. 

	\item {\bf Predict inverse mills ratio} $\widehat{\lambda(\xvar\betab)}
	= \phi(\xvar\check{\betab})/\Phi(\xvar\check{\betab})$.

	\item {\bf Obtain consistent estimates $\widehat{\alpha}$ and
	$\widehat{\gamma}$} by using double selection linear regression of $y_1$
	on $\xvar_1$, $\widehat{\lambda(\xvar\betab)}$, and $\zvar$ as nuisance
	parameter.  For details, see Algorithm \ref{algo:dsregress}.
	\label{algo:dsheck_step4}

	\item {\bf Adjust variance} due to the estimation of $\betab$. For
	details, see Algorithm \ref{algo:adjv}.
	\label{algo:dsheck_step5}
	\end{enumerate}

\end{algorithm}




%------------------------------------------------------------------------------
% algo:dsprobit
%------------------------------------------------------------------------------
\newpage

In Step \ref{algo:dsheck_step2} of Algorithm \ref{algo:dsheckman}, we need to
obtain consistent estimator of $\betab$ and its standard errors.  We apply double
selection for the probit model, which is a special case of \cite{belloni2016}.
Algorithm \ref{algo:dsprobit} describes this procedure.

\bigskip

%------------------------------------------------------------------------------
% algo:dsregress
%------------------------------------------------------------------------------

\newpage

In Step \ref{algo:dsheck_step4} of Algorithm \ref{algo:dsheckman},  we apply
double selection for the linear model to Equation \ref{eq:y1cond2}; see
\cite*{bellonichernozhukovhansen2014}.  Algorithm \ref{algo:dsregress}
describes this procedure.

\bigskip
%------------------------------------------------------------------------------
%	algorithm for dsregress
%------------------------------------------------------------------------------
\begin{algorithm}[H]\label{algo:dsregress}
\caption{Double selection for linear model (vector version of the
double-selection in \cite{belloni:14}}
\begin{description}

	\item{\bf Step 1}. Lasso of $\yvar_1$ on $\zvar$
	\begin{align*}
		\check{\theta_y} \in \argmin \En((\yvar_1 - \zvar'\theta_y)^2) 
		+ \lambda_1 ||\thetab_y||_1
	\end{align*}
	where $\lambda_1$ is chosen by the plugin method described in
	\cite{Drukker/liu:22}.
%

	\item{\bf Step 2}.
	For each variable $m$ in $\xvar_1$ and $\widehat{\lambda(\cdot)}$
	($j=1, \ldots, J$), run lasso of $m$ on $\zvar$.
	\begin{align*}
		\check{\theta_j} \in \argmin \En((m - \zvar'\theta_j)^2) 
		+ \lambda_2 ||\theta_j||_1
	\end{align*}
	where $\lambda_2$ is chosen by the plugin method described in
	\cite{Drukker/liu:22}.
%
	\item{\bf Step 3}.
	OLS of $\yvar_1$ on $\xvar_1$, $\widehat{\lambda(\cdot)}$, and
	$\zvar^*$. $\zvar^*$ is  all the selected variables in Steps 1 and 2.
	\begin{align*}
	(\widehat{\alpha}, \widehat{\gamma}, \widehat{\theta})
	\in
	\argmin \En((\yvar_1 - \xvar_1'\alpha - 
		\widehat{\lambda(\xvar'\check{\betab})}\gamma 
		- {\zvar^*}'\theta)^2)
	\end{align*}
	where $support(\theta) \subseteq support(\check{\theta_y}) \cup 
	support(\check{\theta_1})
	\cup
	\ldots \cup
	support(\check{\theta_J})
	$

	$\widehat{\alpha}$ and $\widehat{\gamma}$ are consistent point
	estimates of $\alpha$ and $\gamma$, but we need to adjust variance
	because $\betab$ is estimated.

\end{description}
\end{algorithm}

%------------------------------------------------------------------------------
% algo: adjust variance
%------------------------------------------------------------------------------

\newpage




We need to adjust variance for $\widehat{\alphab}$ and $\widehat{\gamma}$
obtained in Algorithm \ref{algo:dsregress} because $\widehat{\lambda(\cdot)}$
is estimated. Algorithm \ref{algo:adjv} describes how to adjust variance, which
is basically a implementation of Delta method.
\bigskip
\end{comment}
%------------------------------------------------------------------------------
%	adjust variance
%------------------------------------------------------------------------------
\subsection{Variance matrix estimation}

%\begin{algorithm}[H]\label{algo:adjv}
%\caption{Adjust variance for double-selection Heckman}

We start with some new notation. Let $\hat{\lambda}_i = \widehat{\lambda(\xvar_i'{\betab})}=\lambda(\xvar_i'\check{\betab})$ and define 
\[	\xi_i = \hat{\lambda}_i(\hat{\lambda}_i + \xvar_i' \check{\betab}),
\]
where $\check{\betab}$ is obtained by the double selection probit. %Step 3 of Algorithm
%	\ref{algo:dsprobit}.
%	{\Phi\left( {\xvar'_i\check{\betab}} \right)} 
%\]
%\begin{enumerate}
%	\item The inverse of mills ratio for each observation $i$ is $m_i$
%	$$
%	m_i = \frac{\phi\left( {\xvar\check{\betab}} \right)_i}
%	{\Phi\left( {\xvar\check{\betab}} \right)_i} 
%	$$
%	where $\check{\betab}$ is obtained in Step 3 of Algorithm
%	\ref{algo:dsprobit}.%
%	\item We also define $\xi_i$
%	$$
%	\xi_i = m_i(m_i + \xvar_i' \check{\betab})
%	$$
%	\item Suppose regression the Step 3 of Algorithm \ref{algo:dsregress} is 
Let $e$ denote the vector of residuals from the last step of the double selection lasso estimation, with typical element $e_i, i=1, \ldots, n$. That is, %define 	OLS of $y_1$ on $\xvar_1$, $m_i$, $\zvar^{*}$, where $\zvar^{*}$ is the all the selected variables of $\zvar$.  Compute the regression residuals
\[e_i = y_{1i} - \xvar_{1i}'\widehat{\alpha} - \widehat\lambda_i \widehat{\gamma} -
	{\zvar}_i'\widehat{\theta},\]
where $support(\theta) \subseteq support(\check{\theta}_y) \cup 
	support(\check{\theta}_1)
	\cup
	\ldots \cup
	support(\check{\theta}_J)
$. Let $W$ denote the matrix containing $\xvar_1$, the $n\times 1$ vector of $\hat{\lambda}_i$'s, and the variables in $\zvar$ that survived the double selection. Let $R$ be a $n\times n$ diagonal matrix, with diagonal elements $(1- \hat{\rho}^2\xi_i)$, where 
$\hat{\rho} = \widehat{\gamma}/\widehat{\sigma}$ and 
 $\hat{\sigma}^2 = (e'e + \widehat{\gamma}^2\sum_i \xi_i)/n$.

\begin{prop}\label{prop:var}
A consistent estimator of the variance matrix of the DS-HECK estimator $	(\widehat{\alpha}', \widehat{\gamma}, \widehat{\theta}')$ is 	$$
		V = \hat{\sigma}^2 (W'W)^{-1}(W'RW + Q )(W'W)^{-1},
	$$	where $$
	Q = \hat{\rho}^2 (W' D \xvar) V_b (\xvar'D W)
	$$
	and $V_b$ is the ``sandwich'' variance matrix for the double selection probit estimator $\check{\betab}$ and $D$ is the diagonal matrix with diagonal elements $\xi_i$.
\end{prop}


 The variance for $\widehat{\alpha}$ and $\widehat{\gamma}$ is the
	upper $(k_1+1)\times(k_1+1)$ submatrix of $V$. The {\tt dsheckman} command implements this variance estimator.
%	\end{enumerate}

%\end{algorithm}


%------------------------------------------------------------------------------
\section{Monte Carlo Simulations} \label{sec:sim}
%------------------------------------------------------------------------------

To evaluate the finite-sample performances of DS-HECK, we conduct a simulation study using four estimators: (i) ordinary least squares on the selected sample (\emph{OLS}), (ii) Heckman two-step estimator based on the true model (\emph{Oracle}), (iii) Heckman two-step estimator using lasso to select variables
in Eq.~(\ref{eq:y2}) (\emph{Naive}), and the proposed double selection Heckman estimator (\emph{DS}). We have implemented the DS-HECK estimator in the Stata
command {\tt dsheckman}, available on the authors' web pages along with data sets and codes for the simulations and application, and we describe its syntax in Appendix \ref{sec:dsheckman_syntax}.

\emph{OLS} is inconsistent unless there is no sample selection bias, i.e. $\gamma=0$.  \emph{Naive} is inconsistent due to error made by lasso. Moreover, \emph{Naive} does not provide valid inference as it is not robust to the
model selection bias. In contrast, \emph{DS} is expected to retain consistency in the presence of sample selection biases and show robustness against the model
selection bias. \emph{Oracle} is expected to behave like the standard Heckman
estimator under the true model but, in practice, \emph{Oracle}
is infeasible since we do not know the true model.
%It is interesting to see how the double-selection Heckman estimator behaves
%compared with the oracle, OLS, and naive estimators.
%------------------------------------------------------------------------------
\subsection{Setup}
%------------------------------------------------------------------------------

Our data generating process is as follows
\begin{align*}
y_1 & = 1 + x_1 + x_2 + u_1\\
y_2 &= \mathbb{I} ( \xvar' \beta_0 + \zvar'\eta_0 + u_2 \ge 0), 
\end{align*}
where $u_2 \sim N(0, 1)$ and where $u_1 = \gamma u_2 + \epsilon$ and $\epsilon \sim N(0, 1)$ is independent of $u_2$. We vary the strength  of the selection bias by setting $\gamma$ to be $0.4, 0.5, 0.6, 0.7$, and $0.8$ and we observe $y_1$ only when $y_2 = 1$.

The selection equation is generated using nine non-zero variables in $\zvar$ of which four have a relative large effect and five relatively small:
\begin{align*}
\xvar' \beta_0 + \zvar'\eta_0 &= -1.5 + x_1 - x_2 + z_1 - z_2 + 0.046z_3 + z_5 -
0.046z_{10}  - 0.046z_{11} + 0.046z_{12} - 0.046z_{15} + z_{20}.
\end{align*}
The value $0.046$ is chosen so that it violates the so called ``beta-min'' condition and causes lasso to make model selection mistakes \cite[see, e.g.,][]{liu/etal:20, drukker/liu:22}.  The sample size is $2000$. The number of
replications is $1000$. 

We consider two scenarios for  $p$, the dimension of $\zvar$: (i) $p = 1000$, fewer variables than observations; (ii) $p=2100$, more variables than observations. The variables are generated using a Toeplitz correlation structure with decreasing
dependence. In particular, let $Z$ be the matrix of dimension $N
\times p$ containing $\zvar$, then 
\begin{align*}
Z = M L'
\end{align*} 
where $M$ is $N \times p$ and has the typical element
$(\zeta_{ij} -15)/\sqrt{30}$, where $\zeta_{ij} \sim
\chi^2(15)$ and where $L$ is the Cholesky decomposition of a symmetric Toeplitz
matrix $V$ of dimension $p \times p$ such that its elements obey the following laws: $V_{i,j} =
V_{i-1, j-1}$ and $V_{1, j}= j^{-1.3}$.

The variables $\xvar$ are also correlated and they are generated as functions of $\zvar$. In
particular,
\begin{align*}
x_1 &= z_3 +z_{10} + z_{11} +z_{12} + z_{15} + \epsilon_{x_1} \\
x_2 &= 0.5(z_3 +z_{10} + z_{11} +z_{12} + 2z_{15}) + \epsilon_{x_2}
\end{align*}
where $\epsilon_{x_1}$ and $\epsilon_{x_2}$ follow a Toeplitz structure similar
to $Z$.

As a result, for the selected sample, the true model
is 
\begin{align*}
	y_1 = 1 + x_1 + x_2 + \gamma\lambda(\xvar' \beta_0 + \zvar'\eta_0) + u
	\label{eq:model}
\end{align*}
where $\lambda(\cdot) = \phi(\cdot)/\Phi(\cdot)$ and the true parameter values are
$\beta_1 = \beta_2 = 1$ and $\gamma = 0.4, 0.5, 0.6, 0.7$, or $0.8$.

%------------------------------------------------------------------------------
\subsection{Results}
%------------------------------------------------------------------------------

For each estimator, we report the following measures:
(i) true value of parameter (\emph{True}), (ii) mean squared error (\emph{MSE}),  (iii) average of estimates  across simulations (\emph{Mean}) (iv) standard deviation of estimates across simulations (\emph{SD}), (v) average standard
error across simulations ($\overline{SE}$) and (vi) rejection rate for the $H_0$ that the parameter equals its true value against the nominal 5\% level of significance (\emph{Rej.~rate}).

We report the %\textbf{DELETE: simluation and ADD: simulation} 
simulation results for $\beta_1$, $\beta_2$, and $\gamma$ in Tables \ref{tab:x1}, \ref{tab:x2} and \ref{tab:gamma}, respectively. Several observations are clear from the tables. First, \emph{OLS} is badly biased and the bias is greater when selection is stronger. Second, \emph{Naive} is also biased and it fails to provide valid
inference at any value of $\gamma$ and $p$. This demonstrates that \emph{Naive} is not robust to the model selection errors. The rejection rate and MSE increase as $\gamma$ increases, which is expected
because greater $\gamma$ value indicate a greater sample selection bias. Third, \emph{Oracle} shows consistency and rejection rates close to the nominal 5\% significance level. Fourth, \emph{DS} performs similarly to \emph{Oracle} for all values of $\gamma$ and $p$. In
particular, its MSE is consistently smaller than for \emph{Naive} and \emph{OLS}, 
$\overline{SE}$ is close to \emph{SD}, which shows that the proposed variance adjustment works well. Finally, \emph{Rej.~rate} for \emph{DS} is near the 5\% significance level, which
supports that our estimator offers valid inference.


\begin{center}
\begin{tabular}
[c]{l}\hline\hline
TABLE \ref{tab:x1} ABOUT\ HERE\\\hline\hline
\end{tabular}
\end{center} 

\begin{center}
\begin{tabular}
[c]{l}\hline\hline
TABLE \ref{tab:x2}  ABOUT\ HERE\\\hline\hline
\end{tabular}
\end{center} 

\begin{center}
\begin{tabular}
[c]{l}\hline\hline
TABLE \ref{tab:gamma}  ABOUT\ HERE\\\hline\hline
\end{tabular}
\end{center} 



%------------------------------------------------------------------------------
\section{Application to Female Earnings Estimation} \label{sec:app}
%------------------------------------------------------------------------------
\subsection{Labor force participation and earnings}
Estimation of the female earnings equation is a topic of long standing interest among economists.
%\dil{!!We need fix the citation in this section, right now, the citation are hard coded.}
%have been long studying labor supply behavior of individuals
%including both their probability as well as hours of working and wage rates.
%Originally the focus was on men but later it shifted to women. The lives of
Early investigations of the labor supply decisions including both participation and hours by married women date back to \cite{gronau:74} and \cite{heckman:74}, who were among the first
to highlight sample selection bias stemming from the labor supply decision. % when estimating wage offer regressions. The progress of women in 
Labor market decisions by women over
the later decades have been studied and documented by \cite{Mroz:87, ahn/power:93, neumark/korenman:94, vella:98, devereux:04, blau/kahn:07, mulligan/rubinstein:08, cavalcanti:08, bar/etal:15}, among others.

Numerous applied works have extensively scrutinized the empirical aspects of the
sample selection problem when estimating female labor market behavior. The
determinants of the earnings equations for married women are similar to those of
men and have been mostly agreed upon. These determinants traditionally include
women’s education, experience, age, tenure and location. %, among others. %\footnote{Refer to any of the above mentioned papers for verification.}
The hallmark of correcting for sample selection bias is finding some appropriate
exclusion restriction(s) (i.e., variable(s) affecting selection but not the earnings) % outcome of main interest) 
in order to ensure proper identification of the model
parameters. Two main competing choices of such exclusion restrictions have been
exploited for estimation of labor market earnings for married women:
non-wife/husband’s income and the existence or number of (young) children. The
underlying argument is that these two sets of variables affect the labor supply decision
of married women but not their earnings. \cite{huber/mellace:14}
provide an overview of the related literature on sample selection bias in
the female earnings equations. 

\cite{cavalcanti:08}
provide an alternative view and argue that the declining price and wider
availability of home appliances play a crucial role in explaining the rise in
female labor force participation. This suggests a long list of % leading us to admit that the debate about the best 
potential exclusion restrictions. % can be quite long. %is still ongoing. 
Furthermore, the exact nature and
functional form of the chosen exclusion restriction(s) in the selection
equation is uncertain. %might still be arguably open to further investigation. To give an example, 
For example, %how 
should %one measure 
labor work experience %of an individual  -- in
%be measured in 
include years of %full-time employment or in years that contain 
part-time employment? % as well? how 
Should %one account for the 
educational attainment %of an individual? 
be measured in %In 
full
years of completed education % completed? 
or %Or, perhaps, it is the indicator for 
in millstones 
%educational levels -- 
such as high school or college, % degrees -- that should be
%used instead of (or in addition to) the full years of education completed.
as a replacement or complement to years of education?
Similarly, should age %of an individual 
enter the model in a linear or %form of a liner or
quadratic form? %function? 
%In what follows we attempt to allow the data to speak for
%themselves when answering those questions regarding the determinants of the
%selection equation (as well as their functional forms) when estimating the
%female earnings equation. 

Our %ultimate 
goal in this section is to illustrate the performance of the double-selection Heckman procedure on %an actual empirical setting. To do so we estimate 
the following earnings equation : 
\begin{equation}
\log\left(  earnings\right)
=\alpha_{0}+\alpha_{1}education+\alpha_{2}\mathbf{x}_1+\alpha_{3}state\text{
}dummies+u_1,\label{wreg}
\end{equation}
where $\log\left  (earnings\right)$ is the natural logarithm of the individual's
total annual labor income, $education$ is the person's completed years of
education, $state\text{ }dummies$ is a vector containing a full set of state
dummies, and $u_1$ is an idiosyncratic error.  The vector $\mathbf{x}_1$ varies
across the exact specification we consider % of Eq.~(\ref{wreg}) considered 
and can
contain age and/or work experience. % of the person in either linear or quadratic functional form.  


%We acknowledge 
To address the potential self-selection bias %issue present in the earning equation
%(\ref{wreg}) when estimated as is. Thus, 
we employ DS-HECK %the double-selection Heckman
%approach we propose 
as a data-driven procedure for choosing the explanatory variables and functional
form (among high-dimensional options provided) %for construction of the
%inverse Mills ratio based on %the estimation results from 
in the following
%first-step 
labor force participation equation: 
\begin{equation}
inlf=\mathbf{x}\beta+\mathbf{z}\eta+u_2,\label{probit}
\end{equation}
where $inlf$ is the dummy variable that is equal to one for those women who are
in the labor force at the time of the interview and zero otherwise, and $u_2$ is
an idiosyncratic error. The vector $\mathbf{x}$ includes all the explanatory
variables from Eq.~(\ref{wreg}) (both in a liner and quadratic
functional form) as well as exclusion restrictions.  

In practice, $\mathbf{x}$ is constructed as follows. To simplify notation,
denote all the explanatory variables from Eq.~(\ref{wreg}) as $\mathbf{x}_1$.
First, running lasso probit of $inlf$ on high-dimensional controls $\mathbf{w}$,
where $\mathbf{w}$ includes $\mathbf{x}_1$ and some other high-dimensional
controls.  Denote the selected variables as $\mathbf{x}_2$. Second, $\mathbf{x}$
is union between all $\mathbf{x}_1$ and $\mathbf{x}_2$. All the non-selected
controls in $\mathbf{w}$ are used as $\mathbf{z}$.

%\textbf{Change
%the wording if needed to match the final version of the paper when we get
%there.} 

\subsection{Sample construction}

We obtain our sample from the 2013 wave of the Panel Study of Income Dynamics
(PSID) where we %. To 
focus on the sub-population of
%the high-dimensional variable selection process (rather than the economics behind labor earnings), we employ the data for 
white married women. % only. %[WHAT DOES THAT MEAN?] 
The choice of explanatory variables %in the earnings and selection equations 
reflects their availability in the PSID and %while we generally follow the
the traditional set of regressors %the explanatory variables 
used in the existing literature on
female labor force participation and earnings. Specifically, the explanatory
variables we collect from the PSID include information on the educational
attainment of the individual (both as the number of years completed and as a
set of indicators for milestone achievements in education), a set of indicators
for whether the individual obtained her education in the USA, outside the USA,
or both, as well as a set of indicators for the educational levels of the
individual's parents, work experience of the individual, age and geographical
location of the individual (captured by a set of dummy variables for the current
state where the individual is located), a set of indicators reflecting the
Beale-Ross rural-urban continuum code for the individual's current residence,
and an indicator for whether the individual owns a vehicle. Table \ref{tb:BaselineXs} contains a description of the key explanatory variables.




\begin{center}
\begin{tabular}
[c]{l}\hline\hline
TABLE \ref{tb:BaselineXs} ABOUT\ HERE\\\hline\hline
\end{tabular}
\end{center}

The set of (potential) exclusion restrictions includes the number of children in
the household under 18 years of age, an indicator for whether there are any
children age 15 years old or younger in the individual's household, annual labor
income of the husband, child care expenses, and household major
expenditure (i.e. expenditure on household furnishings and equipment, including
household textiles, furniture, floor coverings, major appliances, small
appliances and miscellaneous housewares). %The latter variable is meant to test
%the argument of Cavalcanti and Tavares (2008) that wider availability of home
%appliances plays a prominent role in explaining the rise in female labor force
%participation. 
While admittedly far from ideal, this last variable is the closest information
we find in the PSID to capture household expenditure on major household
appliances, which allows us to test the argument of \cite{cavalcanti:08}. Finally, the dependent variables for the earnings and selection
equations are (the natural logarithm of) the individual's total annual labor
income and the indicator for whether the individual is in the labor force,
respectively.

Our sample contains 1,989 white married women, of whom 1,294 are in the labor force and 695 are not. Table
\ref{tb:SumStats} reports summary statistics for key variables in the dataset. % used in our analysis. 
A set of dummy variables for the current state %of the individuals 
as well as a
set of indicators reflecting the Beale-Ross rural-urban continuum code %for the individual's current residence 
are omitted to save space.  %from Table \ref{tb:SumStats} but th used as part of the explanatory variables in both the labor income and selection equations. % in our analysis. 
A total of 46 states are present in our sample, with
Delaware, District of Columbia, Hawaii, New Mexico, and Rhode Island omitted %are dropped
from our sample during %the 
data cleaning. % process.
We note that some women report being neither employed nor
(temporarily) unemployed while also reporting non-zero labor income during that
time. %, as reflected in Table \ref{tb:SumStats}. Specifically, 
There are 161 such
women in the sample. %For the purposes of our analysis, w
We treat these
individuals as not being in the labor force. % as they report themselves.


\begin{center}%
\begin{tabular}
[c]{l}\hline\hline
TABLE \ref{tb:SumStats} ABOUT\ HERE\\\hline\hline
\end{tabular}
\end{center} 




\subsection{Empirical findings}

We consider several specifications when estimating the earnings equation subject to sample selection bias. Table \ref{tb:AllEstRes} reports the key results for both equations obtained using DS-HECK. The top panel provides %a subset of 
coefficient estimates (as well as their standard
errors) for $\alpha_1, \alpha_2$ and the coefficient on the inverse Mills ratio
while the bottom panel provides  estimates (and standard errors) for $\beta$. %obtained using the double selection probit  
In addition to the reported estimates, %explanatory variables for each column of Table %
%\ref{tb:ProbitEstRes} described above, all probit specifications in Table \ref{tb:ProbitEstRes} 
each specification contains two more sets of of estimates which we do not report in the table to save space. First, we do not report the coefficients on the full set of state and urban-rural dummies included in both equations. Second, for each specification there are the selected controls in both equations; the number of such controls is reported at the bottom of the respective panels but the coefficients themselves are not reported. % but the relevant coefficients are not reported to save space. %participation additional control variables and a set of
%potential exclusion restrictions. These additional variables/exclusion
%restrictions and their definitions are provided in Table \ref{tb:BaselineXs}.
%Coefficient estimates for the variables included in each probit specification
%that are not reported in Table \ref{tb:ProbitEstRes} are not selected by lasso
%except for the full set of state and urban-rural dummies. We omit the
%coefficient estimates for the latter dummy variables from Table
%\ref{tb:ProbitEstRes} to preserve space (even though these dummy variables are
%among the ones required to be part of the first-step explanatory variables to
%include the explanatory variables from the second step).

We note that following the original Heckman specification, the explanatory variables present in the
earnings equation are always %imposed by our procedure to be 
kept in the labor force participation equation. %its first step regardless of 
%and %whether they are actually selected by 
The lasso selection is not applied to them, only to the additional controls. %in the first step or
%not. 
Columns (1) and (2) %of Table \ref{tb:ProbitEstRes} 
report the estimates %s a subset of the
%first-step estimation results 
when work experience %in a quadratic form enters
%the set of the first-step explanatory variables but only work experience in a
%linear form is present in the labor income equation and thus required to be
%among the ``selected" controls in the first step. 
enters the two equations, with and without the quadratic form, while age is not included. Columns (3) and (4) report the estimate when age is included but experience is not. Columns (5) and (6) contain both but differ in whether experience squared is included. Column (7) reports the estimates of the traditional Heckman specification where no selection is done over the additional controls.   
%is identical to
%column (1) except that work experience in a quadratic form is required to be
%among the ``selected" controls in the probit step. Columns (3) and (4) are
%similar to columns (1) and (2) except that they produce a subset of the
%first-step estimation results when the individual's age (instead of work
%experience as in columns (1)-(2)) is a part of the set of the first-step
%explanatory variables. Columns (4) and (5) follow the logic of the previous two
%pairs of columns but report a subset of the first-step estimation results when
%both age and work experience enter the set of the explanatory variables in the
%first step of our analysis.  

\begin{center}%
\begin{tabular}
[c]{l}\hline\hline
TABLE \ref{tb:AllEstRes} ABOUT\ HERE\\\hline\hline
\end{tabular}
\end{center}

% Finally, column (7) in
%Table \ref{tb:ProbitEstRes} reports a subset of the estimated coefficients from
%the first step probit used in the traditional two-step Heckman sample selection
%procedure where the total number of the explanatory variables from probit
%specification (6) is used as a set of the explanatory variables.




As Table \ref{tb:AllEstRes} suggests, the signs of all the reported coefficient
estimates %of the reported selected explanatory variables 
are as expected, and
they are highly significant for the most part. We note that from the five
potential exclusion restrictions, % offered to lasso to select from, 
the lasso
%method 
selects child care expenditure as the relevant covariate for the labor force participation equation - %. %in our probit
%step. 
this %latter 
variable %plays the role of an  exclusion restriction %in our analysis 
%as it 
is not present in %the %second step of our estimation procedure for
the earnings equation. Finally, we note that the results for the labor force participation equation reported in %first step results in 
column (7) are similar
to those reported in columns (5) and (6) except that the three additional exclusion
restrictions %(if there are children under 15 years old, the number of children, and labor income of the individual's husband) 
turn out to be statistically
significant in the traditional setting. 


Next we focus on the estimates of the labor income equation.% reported in Table \ref{tb:MainEstRes}. 
%The estimation results reported in Table
%\ref{tb:AllEstRes} represent the entire set of explanatory variable
%used in the labor income equation except for a full set of state dummy
%variables as well as a full set of urban-rural %dummy variables. , which are
%omitted to preserve space. 
%Columns (1)-(6) %of Table \ref{tb:MainEstRes} 
%are
%based on the probit specifications (1)-(6) from %respectively. Finally, column (7) reports all the estimated coefficients from
%the second step of the traditional two-step Heckman sample selection procedure
%(except for the state and urban-rural dummies) where probit specification (7)
%from Table \ref{tb:ProbitEstRes} is used as the (traditional) first step. 
%\begin{center}%
%\begin{tabular}
%[c]{l}\hline\hline
%TABLE \ref{tb:MainEstRes} ABOUT\ HERE\\\hline\hline
%\end{tabular}
%\end{center}
According to the results reported in Table \ref{tb:AllEstRes}, we conclude that the educational
level of the individual plays a crucial role in explaining labor income for
white married women in 2012. When statistically significant, the estimated rate
of return to education ranges from 5.6\% to almost 9\% depending on the
specification. Furthermore, there is evidence %we see some indications 
that the individual's age is
more important, both statistically and practically, than work experience for
explaining the individual's labor income in our sample. We note that when the
individual's age (in any functional form) is used in the labor
income equation, the rate of return to education is statistically
significant. 

Most importantly, Table \ref{tb:AllEstRes} suggests that the
inverse Mills ratio is highly statistically significant in all specifications implying that the correction for the sample selection
was needed. %in the labor income equation. 
Given %Having considered 
the economic
interpretation of the estimated coefficients, their signs and economic as well as statistical significance, specification (5) seems most attractive %likely to be the most reliable specification 
in light of the existing studies on
the topic. Interestingly, the traditional Heckman specification produces % second-step estimation 
results that %n column (7) 
are close to those reported in column (5).


%------------------------------------------------------------------------------
\section{Conclusion}	\label{sec:conclude}
%------------------------------------------------------------------------------
We have proposed an extension to the traditional Heckman sample selection model by incorporating it into a model selection framework with many controls. A double application of the lasso to each part of the model permits valid inference on the parts of the model that is of interest to empirical economists. We detail the steps involved in a consistent estimation with valid standard errors and we provide a new Stata command to implement it. We also investigate aspects of the Neyman orthogonality that relate it to the concept of redundancy in the GMM estimation. 

Lasso and double selection in linear models have been recently subject of scrutiny in cases when lasso under-selects controls in finite samples even under a sparsity scenario and the double selection estimators have severe omitted variable biases \cite[see, e.g.,][]{wuthrich/zhu:21, lahiri:21}. This happens when the signal from the variables lasso works on is weak and they do not get selected by either of the two selection procedures. The solution proposed by \cite{wuthrich/zhu:21} is to resort in such cases to the regular OLS estimation using a high-dimensional variance matrix computations which is computationally difficult and works only when $p<n$.  

We show how large the errors committed by the naive application of lasso can be and we provide an application to a classic problem in labor economics where using our method leads to a few new insights. We provide a user-friendly and versatile Stata command, which can help empirical economists use the proposed methodology. The command as well as the simulation and application data are made available on the authors' web pages.

Finally we note that the results of this paper can be extended to other consistent methods of model selection beyond lasso, such as the Dantzig Selector proposed by \cite{candes/tao:07}.
\newpage

%------------------------------------------------------------------------------
%\section{Tables}
%------------------------------------------------------------------------------

\begin{table}[H]
\begin{center}
{\footnotesize
\caption{\textbf{Simulation results for $\beta_1$}}
\label{tab:x1}
\begin{tabular}{llllllllllll}
\cline{1-12}
\multicolumn{1}{c}{} &
  \multicolumn{6}{|c}{$ p = 1000 $} &
  \multicolumn{5}{c}{$ p = 2100 $} \\
\multicolumn{1}{c}{} &
  \multicolumn{1}{|r}{True} &
  \multicolumn{1}{r}{MSE} &
  \multicolumn{1}{r}{Mean} &
  \multicolumn{1}{r}{SD} &
  \multicolumn{1}{r}{$\overline{SE}$} &
  \multicolumn{1}{r}{Rej. Rate} &
  \multicolumn{1}{r}{MSE} &
  \multicolumn{1}{r}{Mean} &
  \multicolumn{1}{r}{SD} &
  \multicolumn{1}{r}{$\overline{SE}$} &
  \multicolumn{1}{r}{Rej. Rate} \\
\cline{1-12}
\multicolumn{1}{l}{$\gamma = 0.4$} &
  \multicolumn{1}{|r}{} &
  \multicolumn{1}{r}{} &
  \multicolumn{1}{r}{} &
  \multicolumn{1}{r}{} &
  \multicolumn{1}{r}{} &
  \multicolumn{1}{r}{} &
  \multicolumn{1}{r}{} &
  \multicolumn{1}{r}{} &
  \multicolumn{1}{r}{} &
  \multicolumn{1}{r}{} &
  \multicolumn{1}{r}{} \\
\multicolumn{1}{l}{\hspace{1em}Oracle} &
  \multicolumn{1}{|r}{1} &
  \multicolumn{1}{r}{0.0014} &
  \multicolumn{1}{r}{0.9984} &
  \multicolumn{1}{r}{0.0369} &
  \multicolumn{1}{r}{0.0374} &
  \multicolumn{1}{r}{0.0450} &
  \multicolumn{1}{r}{0.0013} &
  \multicolumn{1}{r}{0.9985} &
  \multicolumn{1}{r}{0.0362} &
  \multicolumn{1}{r}{0.0374} &
  \multicolumn{1}{r}{0.0510} \\
\multicolumn{1}{l}{\hspace{1em}DS} &
  \multicolumn{1}{|r}{1} &
  \multicolumn{1}{r}{0.0025} &
  \multicolumn{1}{r}{0.9977} &
  \multicolumn{1}{r}{0.0501} &
  \multicolumn{1}{r}{0.0484} &
  \multicolumn{1}{r}{0.0540} &
  \multicolumn{1}{r}{0.0022} &
  \multicolumn{1}{r}{0.9991} &
  \multicolumn{1}{r}{0.0464} &
  \multicolumn{1}{r}{0.0486} &
  \multicolumn{1}{r}{0.0380} \\
\multicolumn{1}{l}{\hspace{1em}Naive} &
  \multicolumn{1}{|r}{1} &
  \multicolumn{1}{r}{0.0049} &
  \multicolumn{1}{r}{0.9424} &
  \multicolumn{1}{r}{0.0395} &
  \multicolumn{1}{r}{0.0350} &
  \multicolumn{1}{r}{0.4200} &
  \multicolumn{1}{r}{0.0052} &
  \multicolumn{1}{r}{0.9379} &
  \multicolumn{1}{r}{0.0360} &
  \multicolumn{1}{r}{0.0349} &
  \multicolumn{1}{r}{0.4310} \\
\multicolumn{1}{l}{\hspace{1em}OLS} &
  \multicolumn{1}{|r}{1} &
  \multicolumn{1}{r}{0.0046} &
  \multicolumn{1}{r}{0.9422} &
  \multicolumn{1}{r}{0.0350} &
  \multicolumn{1}{r}{0.0350} &
  \multicolumn{1}{r}{0.3890} &
  \multicolumn{1}{r}{0.0045} &
  \multicolumn{1}{r}{0.9421} &
  \multicolumn{1}{r}{0.0339} &
  \multicolumn{1}{r}{0.0350} &
  \multicolumn{1}{r}{0.3610} \\
\multicolumn{1}{l}{} &
  \multicolumn{1}{|r}{} &
  \multicolumn{1}{r}{} &
  \multicolumn{1}{r}{} &
  \multicolumn{1}{r}{} &
  \multicolumn{1}{r}{} &
  \multicolumn{1}{r}{} &
  \multicolumn{1}{r}{} &
  \multicolumn{1}{r}{} &
  \multicolumn{1}{r}{} &
  \multicolumn{1}{r}{} &
  \multicolumn{1}{r}{} \\
\multicolumn{1}{l}{$\gamma = 0.5$} &
  \multicolumn{1}{|r}{} &
  \multicolumn{1}{r}{} &
  \multicolumn{1}{r}{} &
  \multicolumn{1}{r}{} &
  \multicolumn{1}{r}{} &
  \multicolumn{1}{r}{} &
  \multicolumn{1}{r}{} &
  \multicolumn{1}{r}{} &
  \multicolumn{1}{r}{} &
  \multicolumn{1}{r}{} &
  \multicolumn{1}{r}{} \\
\multicolumn{1}{l}{\hspace{1em}Oracle} &
  \multicolumn{1}{|r}{1} &
  \multicolumn{1}{r}{0.0014} &
  \multicolumn{1}{r}{0.9993} &
  \multicolumn{1}{r}{0.0380} &
  \multicolumn{1}{r}{0.0387} &
  \multicolumn{1}{r}{0.0430} &
  \multicolumn{1}{r}{0.0015} &
  \multicolumn{1}{r}{0.9980} &
  \multicolumn{1}{r}{0.0389} &
  \multicolumn{1}{r}{0.0387} &
  \multicolumn{1}{r}{0.0480} \\
\multicolumn{1}{l}{\hspace{1em}DS} &
  \multicolumn{1}{|r}{1} &
  \multicolumn{1}{r}{0.0025} &
  \multicolumn{1}{r}{1.0000} &
  \multicolumn{1}{r}{0.0504} &
  \multicolumn{1}{r}{0.0502} &
  \multicolumn{1}{r}{0.0570} &
  \multicolumn{1}{r}{0.0027} &
  \multicolumn{1}{r}{0.9979} &
  \multicolumn{1}{r}{0.0518} &
  \multicolumn{1}{r}{0.0502} &
  \multicolumn{1}{r}{0.0600} \\
\multicolumn{1}{l}{\hspace{1em}Naive} &
  \multicolumn{1}{|r}{1} &
  \multicolumn{1}{r}{0.0067} &
  \multicolumn{1}{r}{0.9289} &
  \multicolumn{1}{r}{0.0409} &
  \multicolumn{1}{r}{0.0362} &
  \multicolumn{1}{r}{0.5240} &
  \multicolumn{1}{r}{0.0075} &
  \multicolumn{1}{r}{0.9229} &
  \multicolumn{1}{r}{0.0389} &
  \multicolumn{1}{r}{0.0361} &
  \multicolumn{1}{r}{0.5840} \\
\multicolumn{1}{l}{\hspace{1em}OLS} &
  \multicolumn{1}{|r}{1} &
  \multicolumn{1}{r}{0.0063} &
  \multicolumn{1}{r}{0.9291} &
  \multicolumn{1}{r}{0.0353} &
  \multicolumn{1}{r}{0.0362} &
  \multicolumn{1}{r}{0.4930} &
  \multicolumn{1}{r}{0.0066} &
  \multicolumn{1}{r}{0.9277} &
  \multicolumn{1}{r}{0.0373} &
  \multicolumn{1}{r}{0.0363} &
  \multicolumn{1}{r}{0.5190} \\
\multicolumn{1}{l}{} &
  \multicolumn{1}{|r}{} &
  \multicolumn{1}{r}{} &
  \multicolumn{1}{r}{} &
  \multicolumn{1}{r}{} &
  \multicolumn{1}{r}{} &
  \multicolumn{1}{r}{} &
  \multicolumn{1}{r}{} &
  \multicolumn{1}{r}{} &
  \multicolumn{1}{r}{} &
  \multicolumn{1}{r}{} &
  \multicolumn{1}{r}{} \\
\multicolumn{1}{l}{$\gamma = 0.6$} &
  \multicolumn{1}{|r}{} &
  \multicolumn{1}{r}{} &
  \multicolumn{1}{r}{} &
  \multicolumn{1}{r}{} &
  \multicolumn{1}{r}{} &
  \multicolumn{1}{r}{} &
  \multicolumn{1}{r}{} &
  \multicolumn{1}{r}{} &
  \multicolumn{1}{r}{} &
  \multicolumn{1}{r}{} &
  \multicolumn{1}{r}{} \\
\multicolumn{1}{l}{\hspace{1em}Oracle} &
  \multicolumn{1}{|r}{1} &
  \multicolumn{1}{r}{0.0016} &
  \multicolumn{1}{r}{0.9995} &
  \multicolumn{1}{r}{0.0395} &
  \multicolumn{1}{r}{0.0401} &
  \multicolumn{1}{r}{0.0440} &
  \multicolumn{1}{r}{0.0015} &
  \multicolumn{1}{r}{0.9984} &
  \multicolumn{1}{r}{0.0393} &
  \multicolumn{1}{r}{0.0400} &
  \multicolumn{1}{r}{0.0520} \\
\multicolumn{1}{l}{\hspace{1em}DS} &
  \multicolumn{1}{|r}{1} &
  \multicolumn{1}{r}{0.0025} &
  \multicolumn{1}{r}{1.0001} &
  \multicolumn{1}{r}{0.0504} &
  \multicolumn{1}{r}{0.0519} &
  \multicolumn{1}{r}{0.0460} &
  \multicolumn{1}{r}{0.0027} &
  \multicolumn{1}{r}{0.9981} &
  \multicolumn{1}{r}{0.0522} &
  \multicolumn{1}{r}{0.0521} &
  \multicolumn{1}{r}{0.0500} \\
\multicolumn{1}{l}{\hspace{1em}Naive} &
  \multicolumn{1}{|r}{1} &
  \multicolumn{1}{r}{0.0089} &
  \multicolumn{1}{r}{0.9183} &
  \multicolumn{1}{r}{0.0475} &
  \multicolumn{1}{r}{0.0376} &
  \multicolumn{1}{r}{0.6110} &
  \multicolumn{1}{r}{0.0101} &
  \multicolumn{1}{r}{0.9081} &
  \multicolumn{1}{r}{0.0413} &
  \multicolumn{1}{r}{0.0373} &
  \multicolumn{1}{r}{0.6880} \\
\multicolumn{1}{l}{\hspace{1em}OLS} &
  \multicolumn{1}{|r}{1} &
  \multicolumn{1}{r}{0.0087} &
  \multicolumn{1}{r}{0.9148} &
  \multicolumn{1}{r}{0.0374} &
  \multicolumn{1}{r}{0.0376} &
  \multicolumn{1}{r}{0.6190} &
  \multicolumn{1}{r}{0.0087} &
  \multicolumn{1}{r}{0.9144} &
  \multicolumn{1}{r}{0.0370} &
  \multicolumn{1}{r}{0.0376} &
  \multicolumn{1}{r}{0.6150} \\
\multicolumn{1}{l}{} &
  \multicolumn{1}{|r}{} &
  \multicolumn{1}{r}{} &
  \multicolumn{1}{r}{} &
  \multicolumn{1}{r}{} &
  \multicolumn{1}{r}{} &
  \multicolumn{1}{r}{} &
  \multicolumn{1}{r}{} &
  \multicolumn{1}{r}{} &
  \multicolumn{1}{r}{} &
  \multicolumn{1}{r}{} &
  \multicolumn{1}{r}{} \\
\multicolumn{1}{l}{$\gamma = 0.7$} &
  \multicolumn{1}{|r}{} &
  \multicolumn{1}{r}{} &
  \multicolumn{1}{r}{} &
  \multicolumn{1}{r}{} &
  \multicolumn{1}{r}{} &
  \multicolumn{1}{r}{} &
  \multicolumn{1}{r}{} &
  \multicolumn{1}{r}{} &
  \multicolumn{1}{r}{} &
  \multicolumn{1}{r}{} &
  \multicolumn{1}{r}{} \\
\multicolumn{1}{l}{\hspace{1em}Oracle} &
  \multicolumn{1}{|r}{1} &
  \multicolumn{1}{r}{0.0019} &
  \multicolumn{1}{r}{0.9969} &
  \multicolumn{1}{r}{0.0435} &
  \multicolumn{1}{r}{0.0416} &
  \multicolumn{1}{r}{0.0690} &
  \multicolumn{1}{r}{0.0018} &
  \multicolumn{1}{r}{1.0001} &
  \multicolumn{1}{r}{0.0419} &
  \multicolumn{1}{r}{0.0419} &
  \multicolumn{1}{r}{0.0510} \\
\multicolumn{1}{l}{\hspace{1em}DS} &
  \multicolumn{1}{|r}{1} &
  \multicolumn{1}{r}{0.0032} &
  \multicolumn{1}{r}{0.9956} &
  \multicolumn{1}{r}{0.0564} &
  \multicolumn{1}{r}{0.0539} &
  \multicolumn{1}{r}{0.0570} &
  \multicolumn{1}{r}{0.0028} &
  \multicolumn{1}{r}{1.0008} &
  \multicolumn{1}{r}{0.0525} &
  \multicolumn{1}{r}{0.0543} &
  \multicolumn{1}{r}{0.0410} \\
\multicolumn{1}{l}{\hspace{1em}Naive} &
  \multicolumn{1}{|r}{1} &
  \multicolumn{1}{r}{0.0126} &
  \multicolumn{1}{r}{0.8999} &
  \multicolumn{1}{r}{0.0508} &
  \multicolumn{1}{r}{0.0389} &
  \multicolumn{1}{r}{0.7400} &
  \multicolumn{1}{r}{0.0135} &
  \multicolumn{1}{r}{0.8930} &
  \multicolumn{1}{r}{0.0455} &
  \multicolumn{1}{r}{0.0390} &
  \multicolumn{1}{r}{0.7640} \\
\multicolumn{1}{l}{\hspace{1em}OLS} &
  \multicolumn{1}{|r}{1} &
  \multicolumn{1}{r}{0.0118} &
  \multicolumn{1}{r}{0.8993} &
  \multicolumn{1}{r}{0.0401} &
  \multicolumn{1}{r}{0.0392} &
  \multicolumn{1}{r}{0.7170} &
  \multicolumn{1}{r}{0.0115} &
  \multicolumn{1}{r}{0.9005} &
  \multicolumn{1}{r}{0.0397} &
  \multicolumn{1}{r}{0.0393} &
  \multicolumn{1}{r}{0.7110} \\
\multicolumn{1}{l}{} &
  \multicolumn{1}{|r}{} &
  \multicolumn{1}{r}{} &
  \multicolumn{1}{r}{} &
  \multicolumn{1}{r}{} &
  \multicolumn{1}{r}{} &
  \multicolumn{1}{r}{} &
  \multicolumn{1}{r}{} &
  \multicolumn{1}{r}{} &
  \multicolumn{1}{r}{} &
  \multicolumn{1}{r}{} &
  \multicolumn{1}{r}{} \\
\multicolumn{1}{l}{$\gamma = 0.8$} &
  \multicolumn{1}{|r}{} &
  \multicolumn{1}{r}{} &
  \multicolumn{1}{r}{} &
  \multicolumn{1}{r}{} &
  \multicolumn{1}{r}{} &
  \multicolumn{1}{r}{} &
  \multicolumn{1}{r}{} &
  \multicolumn{1}{r}{} &
  \multicolumn{1}{r}{} &
  \multicolumn{1}{r}{} &
  \multicolumn{1}{r}{} \\
\multicolumn{1}{l}{\hspace{1em}Oracle} &
  \multicolumn{1}{|r}{1} &
  \multicolumn{1}{r}{0.0019} &
  \multicolumn{1}{r}{0.9962} &
  \multicolumn{1}{r}{0.0437} &
  \multicolumn{1}{r}{0.0434} &
  \multicolumn{1}{r}{0.0470} &
  \multicolumn{1}{r}{0.0019} &
  \multicolumn{1}{r}{0.9968} &
  \multicolumn{1}{r}{0.0438} &
  \multicolumn{1}{r}{0.0434} &
  \multicolumn{1}{r}{0.0510} \\
\multicolumn{1}{l}{\hspace{1em}DS} &
  \multicolumn{1}{|r}{1} &
  \multicolumn{1}{r}{0.0031} &
  \multicolumn{1}{r}{0.9965} &
  \multicolumn{1}{r}{0.0560} &
  \multicolumn{1}{r}{0.0558} &
  \multicolumn{1}{r}{0.0470} &
  \multicolumn{1}{r}{0.0034} &
  \multicolumn{1}{r}{0.9966} &
  \multicolumn{1}{r}{0.0582} &
  \multicolumn{1}{r}{0.0562} &
  \multicolumn{1}{r}{0.0530} \\
\multicolumn{1}{l}{\hspace{1em}Naive} &
  \multicolumn{1}{|r}{1} &
  \multicolumn{1}{r}{0.0165} &
  \multicolumn{1}{r}{0.8835} &
  \multicolumn{1}{r}{0.0544} &
  \multicolumn{1}{r}{0.0406} &
  \multicolumn{1}{r}{0.8020} &
  \multicolumn{1}{r}{0.0173} &
  \multicolumn{1}{r}{0.8777} &
  \multicolumn{1}{r}{0.0484} &
  \multicolumn{1}{r}{0.0404} &
  \multicolumn{1}{r}{0.8360} \\
\multicolumn{1}{l}{\hspace{1em}OLS} &
  \multicolumn{1}{|r}{1} &
  \multicolumn{1}{r}{0.0152} &
  \multicolumn{1}{r}{0.8838} &
  \multicolumn{1}{r}{0.0408} &
  \multicolumn{1}{r}{0.0409} &
  \multicolumn{1}{r}{0.8150} &
  \multicolumn{1}{r}{0.0148} &
  \multicolumn{1}{r}{0.8858} &
  \multicolumn{1}{r}{0.0414} &
  \multicolumn{1}{r}{0.0408} &
  \multicolumn{1}{r}{0.7870} \\
\cline{1-12}
\end{tabular}

}
\end{center}
\end{table}

\begin{table}[H]
\begin{center}
{\footnotesize
\caption{\textbf{Simulation results for $\beta_2$}}
\label{tab:x2}
\begin{tabular}{llllllllllll}
\cline{1-12}
\multicolumn{1}{c}{} &
  \multicolumn{6}{|c}{$ p = 1000 $} &
  \multicolumn{5}{c}{$ p = 2100 $} \\
\multicolumn{1}{c}{} &
  \multicolumn{1}{|r}{True} &
  \multicolumn{1}{r}{MSE} &
  \multicolumn{1}{r}{Mean} &
  \multicolumn{1}{r}{SD} &
  \multicolumn{1}{r}{$\overline{SE}$} &
  \multicolumn{1}{r}{Rej. Rate} &
  \multicolumn{1}{r}{MSE} &
  \multicolumn{1}{r}{Mean} &
  \multicolumn{1}{r}{SD} &
  \multicolumn{1}{r}{$\overline{SE}$} &
  \multicolumn{1}{r}{Rej. Rate} \\
\cline{1-12}
\multicolumn{1}{l}{$\gamma = 0.4$} &
  \multicolumn{1}{|r}{} &
  \multicolumn{1}{r}{} &
  \multicolumn{1}{r}{} &
  \multicolumn{1}{r}{} &
  \multicolumn{1}{r}{} &
  \multicolumn{1}{r}{} &
  \multicolumn{1}{r}{} &
  \multicolumn{1}{r}{} &
  \multicolumn{1}{r}{} &
  \multicolumn{1}{r}{} &
  \multicolumn{1}{r}{} \\
\multicolumn{1}{l}{\hspace{1em}Oracle} &
  \multicolumn{1}{|r}{1} &
  \multicolumn{1}{r}{0.0028} &
  \multicolumn{1}{r}{1.0018} &
  \multicolumn{1}{r}{0.0527} &
  \multicolumn{1}{r}{0.0524} &
  \multicolumn{1}{r}{0.0530} &
  \multicolumn{1}{r}{0.0027} &
  \multicolumn{1}{r}{1.0014} &
  \multicolumn{1}{r}{0.0518} &
  \multicolumn{1}{r}{0.0524} &
  \multicolumn{1}{r}{0.0500} \\
\multicolumn{1}{l}{\hspace{1em}DS} &
  \multicolumn{1}{|r}{1} &
  \multicolumn{1}{r}{0.0033} &
  \multicolumn{1}{r}{1.0015} &
  \multicolumn{1}{r}{0.0576} &
  \multicolumn{1}{r}{0.0584} &
  \multicolumn{1}{r}{0.0390} &
  \multicolumn{1}{r}{0.0034} &
  \multicolumn{1}{r}{1.0025} &
  \multicolumn{1}{r}{0.0580} &
  \multicolumn{1}{r}{0.0587} &
  \multicolumn{1}{r}{0.0530} \\
\multicolumn{1}{l}{\hspace{1em}Naive} &
  \multicolumn{1}{|r}{1} &
  \multicolumn{1}{r}{0.0145} &
  \multicolumn{1}{r}{1.1018} &
  \multicolumn{1}{r}{0.0641} &
  \multicolumn{1}{r}{0.0530} &
  \multicolumn{1}{r}{0.5210} &
  \multicolumn{1}{r}{0.0151} &
  \multicolumn{1}{r}{1.1093} &
  \multicolumn{1}{r}{0.0564} &
  \multicolumn{1}{r}{0.0532} &
  \multicolumn{1}{r}{0.5340} \\
\multicolumn{1}{l}{\hspace{1em}OLS} &
  \multicolumn{1}{|r}{1} &
  \multicolumn{1}{r}{0.0057} &
  \multicolumn{1}{r}{1.0541} &
  \multicolumn{1}{r}{0.0523} &
  \multicolumn{1}{r}{0.0512} &
  \multicolumn{1}{r}{0.1820} &
  \multicolumn{1}{r}{0.0054} &
  \multicolumn{1}{r}{1.0538} &
  \multicolumn{1}{r}{0.0506} &
  \multicolumn{1}{r}{0.0512} &
  \multicolumn{1}{r}{0.1760} \\
\multicolumn{1}{l}{} &
  \multicolumn{1}{|r}{} &
  \multicolumn{1}{r}{} &
  \multicolumn{1}{r}{} &
  \multicolumn{1}{r}{} &
  \multicolumn{1}{r}{} &
  \multicolumn{1}{r}{} &
  \multicolumn{1}{r}{} &
  \multicolumn{1}{r}{} &
  \multicolumn{1}{r}{} &
  \multicolumn{1}{r}{} &
  \multicolumn{1}{r}{} \\
\multicolumn{1}{l}{$\gamma = 0.5$} &
  \multicolumn{1}{|r}{} &
  \multicolumn{1}{r}{} &
  \multicolumn{1}{r}{} &
  \multicolumn{1}{r}{} &
  \multicolumn{1}{r}{} &
  \multicolumn{1}{r}{} &
  \multicolumn{1}{r}{} &
  \multicolumn{1}{r}{} &
  \multicolumn{1}{r}{} &
  \multicolumn{1}{r}{} &
  \multicolumn{1}{r}{} \\
\multicolumn{1}{l}{\hspace{1em}Oracle} &
  \multicolumn{1}{|r}{1} &
  \multicolumn{1}{r}{0.0029} &
  \multicolumn{1}{r}{0.9998} &
  \multicolumn{1}{r}{0.0536} &
  \multicolumn{1}{r}{0.0542} &
  \multicolumn{1}{r}{0.0470} &
  \multicolumn{1}{r}{0.0031} &
  \multicolumn{1}{r}{1.0029} &
  \multicolumn{1}{r}{0.0558} &
  \multicolumn{1}{r}{0.0542} &
  \multicolumn{1}{r}{0.0620} \\
\multicolumn{1}{l}{\hspace{1em}DS} &
  \multicolumn{1}{|r}{1} &
  \multicolumn{1}{r}{0.0036} &
  \multicolumn{1}{r}{1.0005} &
  \multicolumn{1}{r}{0.0600} &
  \multicolumn{1}{r}{0.0603} &
  \multicolumn{1}{r}{0.0380} &
  \multicolumn{1}{r}{0.0038} &
  \multicolumn{1}{r}{1.0030} &
  \multicolumn{1}{r}{0.0613} &
  \multicolumn{1}{r}{0.0606} &
  \multicolumn{1}{r}{0.0600} \\
\multicolumn{1}{l}{\hspace{1em}Naive} &
  \multicolumn{1}{|r}{1} &
  \multicolumn{1}{r}{0.0200} &
  \multicolumn{1}{r}{1.1249} &
  \multicolumn{1}{r}{0.0659} &
  \multicolumn{1}{r}{0.0547} &
  \multicolumn{1}{r}{0.6420} &
  \multicolumn{1}{r}{0.0224} &
  \multicolumn{1}{r}{1.1365} &
  \multicolumn{1}{r}{0.0614} &
  \multicolumn{1}{r}{0.0548} &
  \multicolumn{1}{r}{0.7020} \\
\multicolumn{1}{l}{\hspace{1em}OLS} &
  \multicolumn{1}{|r}{1} &
  \multicolumn{1}{r}{0.0070} &
  \multicolumn{1}{r}{1.0651} &
  \multicolumn{1}{r}{0.0523} &
  \multicolumn{1}{r}{0.0530} &
  \multicolumn{1}{r}{0.2460} &
  \multicolumn{1}{r}{0.0077} &
  \multicolumn{1}{r}{1.0683} &
  \multicolumn{1}{r}{0.0546} &
  \multicolumn{1}{r}{0.0530} &
  \multicolumn{1}{r}{0.2540} \\
\multicolumn{1}{l}{} &
  \multicolumn{1}{|r}{} &
  \multicolumn{1}{r}{} &
  \multicolumn{1}{r}{} &
  \multicolumn{1}{r}{} &
  \multicolumn{1}{r}{} &
  \multicolumn{1}{r}{} &
  \multicolumn{1}{r}{} &
  \multicolumn{1}{r}{} &
  \multicolumn{1}{r}{} &
  \multicolumn{1}{r}{} &
  \multicolumn{1}{r}{} \\
\multicolumn{1}{l}{$\gamma = 0.6$} &
  \multicolumn{1}{|r}{} &
  \multicolumn{1}{r}{} &
  \multicolumn{1}{r}{} &
  \multicolumn{1}{r}{} &
  \multicolumn{1}{r}{} &
  \multicolumn{1}{r}{} &
  \multicolumn{1}{r}{} &
  \multicolumn{1}{r}{} &
  \multicolumn{1}{r}{} &
  \multicolumn{1}{r}{} &
  \multicolumn{1}{r}{} \\
\multicolumn{1}{l}{\hspace{1em}Oracle} &
  \multicolumn{1}{|r}{1} &
  \multicolumn{1}{r}{0.0031} &
  \multicolumn{1}{r}{1.0014} &
  \multicolumn{1}{r}{0.0555} &
  \multicolumn{1}{r}{0.0562} &
  \multicolumn{1}{r}{0.0510} &
  \multicolumn{1}{r}{0.0030} &
  \multicolumn{1}{r}{1.0020} &
  \multicolumn{1}{r}{0.0548} &
  \multicolumn{1}{r}{0.0560} &
  \multicolumn{1}{r}{0.0450} \\
\multicolumn{1}{l}{\hspace{1em}DS} &
  \multicolumn{1}{|r}{1} &
  \multicolumn{1}{r}{0.0038} &
  \multicolumn{1}{r}{1.0034} &
  \multicolumn{1}{r}{0.0620} &
  \multicolumn{1}{r}{0.0623} &
  \multicolumn{1}{r}{0.0450} &
  \multicolumn{1}{r}{0.0038} &
  \multicolumn{1}{r}{1.0038} &
  \multicolumn{1}{r}{0.0615} &
  \multicolumn{1}{r}{0.0628} &
  \multicolumn{1}{r}{0.0410} \\
\multicolumn{1}{l}{\hspace{1em}Naive} &
  \multicolumn{1}{|r}{1} &
  \multicolumn{1}{r}{0.0271} &
  \multicolumn{1}{r}{1.1453} &
  \multicolumn{1}{r}{0.0775} &
  \multicolumn{1}{r}{0.0566} &
  \multicolumn{1}{r}{0.7150} &
  \multicolumn{1}{r}{0.0307} &
  \multicolumn{1}{r}{1.1626} &
  \multicolumn{1}{r}{0.0653} &
  \multicolumn{1}{r}{0.0567} &
  \multicolumn{1}{r}{0.8000} \\
\multicolumn{1}{l}{\hspace{1em}OLS} &
  \multicolumn{1}{|r}{1} &
  \multicolumn{1}{r}{0.0095} &
  \multicolumn{1}{r}{1.0804} &
  \multicolumn{1}{r}{0.0550} &
  \multicolumn{1}{r}{0.0550} &
  \multicolumn{1}{r}{0.3100} &
  \multicolumn{1}{r}{0.0092} &
  \multicolumn{1}{r}{1.0795} &
  \multicolumn{1}{r}{0.0536} &
  \multicolumn{1}{r}{0.0549} &
  \multicolumn{1}{r}{0.2950} \\
\multicolumn{1}{l}{} &
  \multicolumn{1}{|r}{} &
  \multicolumn{1}{r}{} &
  \multicolumn{1}{r}{} &
  \multicolumn{1}{r}{} &
  \multicolumn{1}{r}{} &
  \multicolumn{1}{r}{} &
  \multicolumn{1}{r}{} &
  \multicolumn{1}{r}{} &
  \multicolumn{1}{r}{} &
  \multicolumn{1}{r}{} &
  \multicolumn{1}{r}{} \\
\multicolumn{1}{l}{$\gamma = 0.7$} &
  \multicolumn{1}{|r}{} &
  \multicolumn{1}{r}{} &
  \multicolumn{1}{r}{} &
  \multicolumn{1}{r}{} &
  \multicolumn{1}{r}{} &
  \multicolumn{1}{r}{} &
  \multicolumn{1}{r}{} &
  \multicolumn{1}{r}{} &
  \multicolumn{1}{r}{} &
  \multicolumn{1}{r}{} &
  \multicolumn{1}{r}{} \\
\multicolumn{1}{l}{\hspace{1em}Oracle} &
  \multicolumn{1}{|r}{1} &
  \multicolumn{1}{r}{0.0036} &
  \multicolumn{1}{r}{1.0031} &
  \multicolumn{1}{r}{0.0596} &
  \multicolumn{1}{r}{0.0583} &
  \multicolumn{1}{r}{0.0620} &
  \multicolumn{1}{r}{0.0034} &
  \multicolumn{1}{r}{0.9997} &
  \multicolumn{1}{r}{0.0582} &
  \multicolumn{1}{r}{0.0585} &
  \multicolumn{1}{r}{0.0430} \\
\multicolumn{1}{l}{\hspace{1em}DS} &
  \multicolumn{1}{|r}{1} &
  \multicolumn{1}{r}{0.0042} &
  \multicolumn{1}{r}{1.0034} &
  \multicolumn{1}{r}{0.0649} &
  \multicolumn{1}{r}{0.0648} &
  \multicolumn{1}{r}{0.0590} &
  \multicolumn{1}{r}{0.0043} &
  \multicolumn{1}{r}{0.9994} &
  \multicolumn{1}{r}{0.0653} &
  \multicolumn{1}{r}{0.0657} &
  \multicolumn{1}{r}{0.0430} \\
\multicolumn{1}{l}{\hspace{1em}Naive} &
  \multicolumn{1}{|r}{1} &
  \multicolumn{1}{r}{0.0377} &
  \multicolumn{1}{r}{1.1755} &
  \multicolumn{1}{r}{0.0833} &
  \multicolumn{1}{r}{0.0587} &
  \multicolumn{1}{r}{0.8170} &
  \multicolumn{1}{r}{0.0414} &
  \multicolumn{1}{r}{1.1902} &
  \multicolumn{1}{r}{0.0725} &
  \multicolumn{1}{r}{0.0591} &
  \multicolumn{1}{r}{0.8680} \\
\multicolumn{1}{l}{\hspace{1em}OLS} &
  \multicolumn{1}{|r}{1} &
  \multicolumn{1}{r}{0.0121} &
  \multicolumn{1}{r}{1.0934} &
  \multicolumn{1}{r}{0.0583} &
  \multicolumn{1}{r}{0.0573} &
  \multicolumn{1}{r}{0.3740} &
  \multicolumn{1}{r}{0.0118} &
  \multicolumn{1}{r}{1.0925} &
  \multicolumn{1}{r}{0.0572} &
  \multicolumn{1}{r}{0.0575} &
  \multicolumn{1}{r}{0.3640} \\
\multicolumn{1}{l}{} &
  \multicolumn{1}{|r}{} &
  \multicolumn{1}{r}{} &
  \multicolumn{1}{r}{} &
  \multicolumn{1}{r}{} &
  \multicolumn{1}{r}{} &
  \multicolumn{1}{r}{} &
  \multicolumn{1}{r}{} &
  \multicolumn{1}{r}{} &
  \multicolumn{1}{r}{} &
  \multicolumn{1}{r}{} &
  \multicolumn{1}{r}{} \\
\multicolumn{1}{l}{$\gamma = 0.8$} &
  \multicolumn{1}{|r}{} &
  \multicolumn{1}{r}{} &
  \multicolumn{1}{r}{} &
  \multicolumn{1}{r}{} &
  \multicolumn{1}{r}{} &
  \multicolumn{1}{r}{} &
  \multicolumn{1}{r}{} &
  \multicolumn{1}{r}{} &
  \multicolumn{1}{r}{} &
  \multicolumn{1}{r}{} &
  \multicolumn{1}{r}{} \\
\multicolumn{1}{l}{\hspace{1em}Oracle} &
  \multicolumn{1}{|r}{1} &
  \multicolumn{1}{r}{0.0037} &
  \multicolumn{1}{r}{1.0060} &
  \multicolumn{1}{r}{0.0606} &
  \multicolumn{1}{r}{0.0607} &
  \multicolumn{1}{r}{0.0440} &
  \multicolumn{1}{r}{0.0037} &
  \multicolumn{1}{r}{1.0016} &
  \multicolumn{1}{r}{0.0611} &
  \multicolumn{1}{r}{0.0608} &
  \multicolumn{1}{r}{0.0520} \\
\multicolumn{1}{l}{\hspace{1em}DS} &
  \multicolumn{1}{|r}{1} &
  \multicolumn{1}{r}{0.0049} &
  \multicolumn{1}{r}{1.0064} &
  \multicolumn{1}{r}{0.0698} &
  \multicolumn{1}{r}{0.0676} &
  \multicolumn{1}{r}{0.0630} &
  \multicolumn{1}{r}{0.0049} &
  \multicolumn{1}{r}{1.0010} &
  \multicolumn{1}{r}{0.0703} &
  \multicolumn{1}{r}{0.0683} &
  \multicolumn{1}{r}{0.0530} \\
\multicolumn{1}{l}{\hspace{1em}Naive} &
  \multicolumn{1}{|r}{1} &
  \multicolumn{1}{r}{0.0505} &
  \multicolumn{1}{r}{1.2059} &
  \multicolumn{1}{r}{0.0898} &
  \multicolumn{1}{r}{0.0611} &
  \multicolumn{1}{r}{0.8690} &
  \multicolumn{1}{r}{0.0519} &
  \multicolumn{1}{r}{1.2143} &
  \multicolumn{1}{r}{0.0777} &
  \multicolumn{1}{r}{0.0613} &
  \multicolumn{1}{r}{0.8990} \\
\multicolumn{1}{l}{\hspace{1em}OLS} &
  \multicolumn{1}{|r}{1} &
  \multicolumn{1}{r}{0.0158} &
  \multicolumn{1}{r}{1.1106} &
  \multicolumn{1}{r}{0.0598} &
  \multicolumn{1}{r}{0.0598} &
  \multicolumn{1}{r}{0.4510} &
  \multicolumn{1}{r}{0.0147} &
  \multicolumn{1}{r}{1.1052} &
  \multicolumn{1}{r}{0.0600} &
  \multicolumn{1}{r}{0.0598} &
  \multicolumn{1}{r}{0.4250} \\
\cline{1-12}
\end{tabular}

}
\end{center}
\end{table}

\begin{table}[H]
\begin{center}
{\footnotesize
\caption{\textbf{Simulation results for $\gamma$}}
\label{tab:gamma}
\begin{tabular}{llllllllllll}
\cline{1-12}
\multicolumn{1}{c}{} &
  \multicolumn{6}{|c}{$ p = 1000 $} &
  \multicolumn{5}{c}{$ p = 2100 $} \\
\multicolumn{1}{c}{} &
  \multicolumn{1}{|r}{True} &
  \multicolumn{1}{r}{MSE} &
  \multicolumn{1}{r}{Mean} &
  \multicolumn{1}{r}{SD} &
  \multicolumn{1}{r}{$\overline{SE}$} &
  \multicolumn{1}{r}{Rej. Rate} &
  \multicolumn{1}{r}{MSE} &
  \multicolumn{1}{r}{Mean} &
  \multicolumn{1}{r}{SD} &
  \multicolumn{1}{r}{$\overline{SE}$} &
  \multicolumn{1}{r}{Rej. Rate} \\
\cline{1-12}
%\multicolumn{1}{l}{$\gamma = 0.4$} &
%  \multicolumn{1}{|r}{} &
%  \multicolumn{1}{r}{} &
%%  \multicolumn{1}{r}{} &
 % \multicolumn{1}{r}{} &
%  \multicolumn{1}{r}{} &
%  \multicolumn{1}{r}{} &
%  \multicolumn{1}{r}{} &
%  \multicolumn{1}{r}{} &
%  \multicolumn{1}{r}{} &
%  \multicolumn{1}{r}{} &
%  \multicolumn{1}{r}{} \\
\multicolumn{1}{l}{\hspace{1em}Oracle} &
  \multicolumn{1}{|r}{0.4} &
  \multicolumn{1}{r}{0.0092} &
  \multicolumn{1}{r}{0.3948} &
  \multicolumn{1}{r}{0.0959} &
  \multicolumn{1}{r}{0.0948} &
  \multicolumn{1}{r}{0.0490} &
  \multicolumn{1}{r}{0.0087} &
  \multicolumn{1}{r}{0.3969} &
  \multicolumn{1}{r}{0.0933} &
  \multicolumn{1}{r}{0.0952} &
  \multicolumn{1}{r}{0.0430} \\
\multicolumn{1}{l}{\hspace{1em}DS} &
  \multicolumn{1}{|r}{0.4} &
  \multicolumn{1}{r}{0.0096} &
  \multicolumn{1}{r}{0.3873} &
  \multicolumn{1}{r}{0.0970} &
  \multicolumn{1}{r}{0.0943} &
  \multicolumn{1}{r}{0.0580} &
  \multicolumn{1}{r}{0.0091} &
  \multicolumn{1}{r}{0.3877} &
  \multicolumn{1}{r}{0.0946} &
  \multicolumn{1}{r}{0.0942} &
  \multicolumn{1}{r}{0.0510} \\
\multicolumn{1}{l}{\hspace{1em}Naive} &
  \multicolumn{1}{|r}{0.4} &
  \multicolumn{1}{r}{0.0115} &
  \multicolumn{1}{r}{0.3658} &
  \multicolumn{1}{r}{0.1017} &
  \multicolumn{1}{r}{0.0982} &
  \multicolumn{1}{r}{0.0710} &
  \multicolumn{1}{r}{0.0108} &
  \multicolumn{1}{r}{0.3646} &
  \multicolumn{1}{r}{0.0979} &
  \multicolumn{1}{r}{0.0988} &
  \multicolumn{1}{r}{0.0590} \\
\multicolumn{1}{l}{} &
  \multicolumn{1}{|r}{} &
  \multicolumn{1}{r}{} &
  \multicolumn{1}{r}{} &
  \multicolumn{1}{r}{} &
  \multicolumn{1}{r}{} &
  \multicolumn{1}{r}{} &
  \multicolumn{1}{r}{} &
  \multicolumn{1}{r}{} &
  \multicolumn{1}{r}{} &
  \multicolumn{1}{r}{} &
  \multicolumn{1}{r}{} \\
%\multicolumn{1}{l}{$\gamma = 0.5$} &
%  \multicolumn{1}{|r}{} &
%  \multicolumn{1}{r}{} &
%  \multicolumn{1}{r}{} &
%  \multicolumn{1}{r}{} &
%  \multicolumn{1}{r}{} &
%  \multicolumn{1}{r}{} &
%  \multicolumn{1}{r}{} &
%  \multicolumn{1}{r}{} &
%  \multicolumn{1}{r}{} &
%  \multicolumn{1}{r}{} &
%  \multicolumn{1}{r}{} \\
\multicolumn{1}{l}{\hspace{1em}Oracle} &
  \multicolumn{1}{|r}{0.5} &
  \multicolumn{1}{r}{0.0102} &
  \multicolumn{1}{r}{0.4940} &
  \multicolumn{1}{r}{0.1011} &
  \multicolumn{1}{r}{0.0973} &
  \multicolumn{1}{r}{0.0540} &
  \multicolumn{1}{r}{0.0093} &
  \multicolumn{1}{r}{0.4948} &
  \multicolumn{1}{r}{0.0961} &
  \multicolumn{1}{r}{0.0972} &
  \multicolumn{1}{r}{0.0460} \\
\multicolumn{1}{l}{\hspace{1em}DS} &
  \multicolumn{1}{|r}{.05} &
  \multicolumn{1}{r}{0.0107} &
  \multicolumn{1}{r}{0.4843} &
  \multicolumn{1}{r}{0.1022} &
  \multicolumn{1}{r}{0.0969} &
  \multicolumn{1}{r}{0.0650} &
  \multicolumn{1}{r}{0.0095} &
  \multicolumn{1}{r}{0.4837} &
  \multicolumn{1}{r}{0.0963} &
  \multicolumn{1}{r}{0.0964} &
  \multicolumn{1}{r}{0.0550} \\
\multicolumn{1}{l}{\hspace{1em}Naive} &
  \multicolumn{1}{|r}{0.5} &
  \multicolumn{1}{r}{0.0135} &
  \multicolumn{1}{r}{0.4558} &
  \multicolumn{1}{r}{0.1077} &
  \multicolumn{1}{r}{0.1010} &
  \multicolumn{1}{r}{0.0820} &
  \multicolumn{1}{r}{0.0125} &
  \multicolumn{1}{r}{0.4537} &
  \multicolumn{1}{r}{0.1017} &
  \multicolumn{1}{r}{0.1013} &
  \multicolumn{1}{r}{0.0760} \\
\multicolumn{1}{l}{} &
  \multicolumn{1}{|r}{} &
  \multicolumn{1}{r}{} &
  \multicolumn{1}{r}{} &
  \multicolumn{1}{r}{} &
  \multicolumn{1}{r}{} &
  \multicolumn{1}{r}{} &
  \multicolumn{1}{r}{} &
  \multicolumn{1}{r}{} &
  \multicolumn{1}{r}{} &
  \multicolumn{1}{r}{} &
  \multicolumn{1}{r}{} \\
%\multicolumn{1}{l}{$\gamma = 0.6$} &
%  \multicolumn{1}{|r}{} &
%  \multicolumn{1}{r}{} &
%  \multicolumn{1}{r}{} &
%  \multicolumn{1}{r}{} &
%  \multicolumn{1}{r}{} &
%  \multicolumn{1}{r}{} &
%  \multicolumn{1}{r}{} &
%  \multicolumn{1}{r}{} &
%  \multicolumn{1}{r}{} &
%  \multicolumn{1}{r}{} &
%  \multicolumn{1}{r}{} \\
\multicolumn{1}{l}{\hspace{1em}Oracle} &
  \multicolumn{1}{|r}{0.6} &
  \multicolumn{1}{r}{0.0092} &
  \multicolumn{1}{r}{0.5952} &
  \multicolumn{1}{r}{0.0958} &
  \multicolumn{1}{r}{0.1000} &
  \multicolumn{1}{r}{0.0390} &
  \multicolumn{1}{r}{0.0098} &
  \multicolumn{1}{r}{0.5945} &
  \multicolumn{1}{r}{0.0988} &
  \multicolumn{1}{r}{0.1000} &
  \multicolumn{1}{r}{0.0460} \\
\multicolumn{1}{l}{\hspace{1em}DS} &
  \multicolumn{1}{|r}{0.6} &
  \multicolumn{1}{r}{0.0098} &
  \multicolumn{1}{r}{0.5835} &
  \multicolumn{1}{r}{0.0977} &
  \multicolumn{1}{r}{0.0998} &
  \multicolumn{1}{r}{0.0420} &
  \multicolumn{1}{r}{0.0102} &
  \multicolumn{1}{r}{0.5810} &
  \multicolumn{1}{r}{0.0995} &
  \multicolumn{1}{r}{0.0993} &
  \multicolumn{1}{r}{0.0500} \\
\multicolumn{1}{l}{\hspace{1em}Naive} &
  \multicolumn{1}{|r}{0.6} &
  \multicolumn{1}{r}{0.0135} &
  \multicolumn{1}{r}{0.5454} &
  \multicolumn{1}{r}{0.1024} &
  \multicolumn{1}{r}{0.1041} &
  \multicolumn{1}{r}{0.0780} &
  \multicolumn{1}{r}{0.0142} &
  \multicolumn{1}{r}{0.5450} &
  \multicolumn{1}{r}{0.1058} &
  \multicolumn{1}{r}{0.1050} &
  \multicolumn{1}{r}{0.0800} \\
\multicolumn{1}{l}{} &
  \multicolumn{1}{|r}{} &
  \multicolumn{1}{r}{} &
  \multicolumn{1}{r}{} &
  \multicolumn{1}{r}{} &
  \multicolumn{1}{r}{} &
  \multicolumn{1}{r}{} &
  \multicolumn{1}{r}{} &
  \multicolumn{1}{r}{} &
  \multicolumn{1}{r}{} &
  \multicolumn{1}{r}{} &
  \multicolumn{1}{r}{} \\
%\multicolumn{1}{l}{$\gamma = .7$} &
%  \multicolumn{1}{|r}{} &
%  \multicolumn{1}{r}{} &
%  \multicolumn{1}{r}{} &
%  \multicolumn{1}{r}{} &
%  \multicolumn{1}{r}{} &
%  \multicolumn{1}{r}{} &
%  \multicolumn{1}{r}{} &
%  \multicolumn{1}{r}{} &
%  \multicolumn{1}{r}{} &
%  \multicolumn{1}{r}{} &
%  \multicolumn{1}{r}{} \\
\multicolumn{1}{l}{\hspace{1em}Oracle} &
  \multicolumn{1}{|r}{0.7} &
  \multicolumn{1}{r}{0.0116} &
  \multicolumn{1}{r}{0.6903} &
  \multicolumn{1}{r}{0.1072} &
  \multicolumn{1}{r}{0.1030} &
  \multicolumn{1}{r}{0.0600} &
  \multicolumn{1}{r}{0.0104} &
  \multicolumn{1}{r}{0.6999} &
  \multicolumn{1}{r}{0.1020} &
  \multicolumn{1}{r}{0.1031} &
  \multicolumn{1}{r}{0.0410} \\
\multicolumn{1}{l}{\hspace{1em}DS} &
  \multicolumn{1}{|r}{0.7} &
  \multicolumn{1}{r}{0.0124} &
  \multicolumn{1}{r}{0.6765} &
  \multicolumn{1}{r}{0.1091} &
  \multicolumn{1}{r}{0.1029} &
  \multicolumn{1}{r}{0.0770} &
  \multicolumn{1}{r}{0.0111} &
  \multicolumn{1}{r}{0.6830} &
  \multicolumn{1}{r}{0.1040} &
  \multicolumn{1}{r}{0.1025} &
  \multicolumn{1}{r}{0.0510} \\
\multicolumn{1}{l}{\hspace{1em}Naive} &
  \multicolumn{1}{|r}{0.7} &
  \multicolumn{1}{r}{0.0172} &
  \multicolumn{1}{r}{0.6359} &
  \multicolumn{1}{r}{0.1145} &
  \multicolumn{1}{r}{0.1080} &
  \multicolumn{1}{r}{0.1170} &
  \multicolumn{1}{r}{0.0153} &
  \multicolumn{1}{r}{0.6453} &
  \multicolumn{1}{r}{0.1111} &
  \multicolumn{1}{r}{0.1091} &
  \multicolumn{1}{r}{0.0740} \\
\multicolumn{1}{l}{} &
  \multicolumn{1}{|r}{} &
  \multicolumn{1}{r}{} &
  \multicolumn{1}{r}{} &
  \multicolumn{1}{r}{} &
  \multicolumn{1}{r}{} &
  \multicolumn{1}{r}{} &
  \multicolumn{1}{r}{} &
  \multicolumn{1}{r}{} &
  \multicolumn{1}{r}{} &
  \multicolumn{1}{r}{} &
  \multicolumn{1}{r}{} \\
%\multicolumn{1}{l}{$\gamma = .8$} &
%  \multicolumn{1}{|r}{} &
%  \multicolumn{1}{r}{} &
%  \multicolumn{1}{r}{} &
%  \multicolumn{1}{r}{} &
%  \multicolumn{1}{r}{} &
%  \multicolumn{1}{r}{} &
%  \multicolumn{1}{r}{} &
%  \multicolumn{1}{r}{} &
%  \multicolumn{1}{r}{} &
%  \multicolumn{1}{r}{} &
%  \multicolumn{1}{r}{} \\
\multicolumn{1}{l}{\hspace{1em}Oracle} &
  \multicolumn{1}{|r}{0.8} &
  \multicolumn{1}{r}{0.0117} &
  \multicolumn{1}{r}{0.7912} &
  \multicolumn{1}{r}{0.1078} &
  \multicolumn{1}{r}{0.1064} &
  \multicolumn{1}{r}{0.0510} &
  \multicolumn{1}{r}{0.0117} &
  \multicolumn{1}{r}{0.7831} &
  \multicolumn{1}{r}{0.1071} &
  \multicolumn{1}{r}{0.1066} &
  \multicolumn{1}{r}{0.0560} \\
\multicolumn{1}{l}{\hspace{1em}DS} &
  \multicolumn{1}{|r}{0.8} &
  \multicolumn{1}{r}{0.0131} &
  \multicolumn{1}{r}{0.7762} &
  \multicolumn{1}{r}{0.1119} &
  \multicolumn{1}{r}{0.1065} &
  \multicolumn{1}{r}{0.0720} &
  \multicolumn{1}{r}{0.0130} &
  \multicolumn{1}{r}{0.7657} &
  \multicolumn{1}{r}{0.1088} &
  \multicolumn{1}{r}{0.1062} &
  \multicolumn{1}{r}{0.0820} \\
\multicolumn{1}{l}{\hspace{1em}Naive} &
  \multicolumn{1}{|r}{0.8} &
  \multicolumn{1}{r}{0.0200} &
  \multicolumn{1}{r}{0.7260} &
  \multicolumn{1}{r}{0.1204} &
  \multicolumn{1}{r}{0.1124} &
  \multicolumn{1}{r}{0.1380} &
  \multicolumn{1}{r}{0.0202} &
  \multicolumn{1}{r}{0.7213} &
  \multicolumn{1}{r}{0.1182} &
  \multicolumn{1}{r}{0.1132} &
  \multicolumn{1}{r}{0.1240} \\
\cline{1-12}
\end{tabular}

}
\end{center}
\end{table}

\newpage

\begin{flushleft}

\begin{table}[ht]
\caption{\textbf{Description of key PSID variables%Variables' Types \& Their Definitions for the First Step
}
\label{tb:BaselineXs}}
%\begin{center}
\resizebox{0.8\textwidth}{!}{\begin{minipage}{\textwidth}
\begin{tabular}{ p{5cm}|p{3cm}|p{12cm}  }
 \hline \hline
%\centering 
Variable& \centering Type & Definition \\
\hline\hline
\multicolumn{3}{c}{\textbf{Controls}}\\
\hline
%\centering 
Education & \centering Continuous & Years of education\\
%\centering 
%Educational milestones & \centering Categorical  &  Educational milestone achievements\\
High school education & \centering Categorical & Graduated from high school (3 categories)\\
Enrolled in school & \centering Categorical & If currently enrolled in regular school (2 categories)\\
College attendance & \centering Categorical & If attended college (2 categories) \\
Other degree or certificate & \centering Categorical & If received other degree/certificate (2 categories) \\
%\centering 
US education & \centering Categorical & If the individual obtained her education in the USA, outside the USA, or both (3 categories)\\%
%\centering 
Father's education & \centering Categorical & Educational level of the individual's father (8 categories) \\
%\centering 
Mother's education & \centering Categorical & Educational level of the individual's mother (8 categories)\\
%\centering 
If owns a vehicle & \centering Categorical & If the individual owns a vehicle (2 categories) \\
%\centering 
Current state & \centering Categorical & Geographical location of the individual (46 states) \\
%\centering 
Rural-urban location & \centering Categorical & Beale-Ross rural-urban continuum code for the individual's current residence (9 categories) \\
\hline
\multicolumn{3}{c}{\textbf{Potential exclusion restrictions}}\\
\hline
%\centering 
Number of kids & \centering Continuous & The number of children in the household under 18 years of age \\
%\centering 
If kids less than or equal to 15 years old & \centering Categorical & If there are any children age 15 years old or younger in the individual's household (2 categories)\\
%\centering 
Child care expenditure & \centering Continuous & Child care expenses (in thousand dollars)\\
%\centering 
Husband's labor income & \centering Continuous & Annual labor income of the individual's husband (in thousand dollas) \\
%\centering 
Household major expenditure & \centering Continuous & Expenses on household furnishings and equipment, including household textiles, furniture, floor coverings, major appliances, small appliances and miscellaneous housewares (in thousand dollars)\\
\hline \hline
\end{tabular}
\bigskip
\bigskip
\end{minipage}}
\begin{center}
\vspace{-.5cm}
%\begin{minipage}{16.8cm}%
%    \footnotesize  \textbf{Notes:} This table lists and describes the main variables that are employed in our analysis. Descriptive statistics for most of these variables are provided in Table \ref{tb:SumStats}. %Please refer to the main text for further details.  %
 % \end{minipage}%
  \end{center}
\end{table}

\end{flushleft}


\newpage
\begin{center}
{\footnotesize 
\begin{longtable}{lccccccc}
\caption{\textbf{Key sample characteristics}}\label{tb:SumStats}
\vspace{.05in}
\\
\hline \hline
            \multicolumn{1}{c}{} &\multicolumn{2}{c}{In the labor force} &\multicolumn{2}{c}{Not in the labor force} & \multicolumn{2}{c}{All}\\
%\textbf{Variables} 
&        Mean&          SD&        Mean&          SD&        Mean&          SD \\
\hline \hline
\\
\textbf{Controls}\\
\hline
Age         &       43.84&       11.74&       53.63&       16.17&       47.26&       14.24 \\
Experience  &       9.86&       6.64&       12.21&       9.30&       10.68&       7.75\\
Education (in years)&       14.78&       1.98&       13.96&       2.16&       14.49&       2.08\\
Graduated from high school (1-3 range) & 1.05 & 0.27 &  1.14 & 0.46 & 1.09 & 0.35 \\
If currently enrolled in regular school &     0.03  &  0.17  & 0.03   & 0.16 &   0.03  & 0.17 \\
If attended college  &     0.82  &  0.39   & 0.68  & 0.47 & 0.77 & 0.42  \\
If received other degree/certificate  &    0.21  &  0.41  & 0.20 & 0.40 & 0.20 & 0.40 \\
Received  education in the USA  (1-3 range) &     1.04  &   0.30  & 1.04  & 0.27 & 1.04 & 0.29 \\
Father's educational level  (1-8 range) &    5.08   & 1.80  & 4.56 & 1.92 & 4.90 & 1.86 \\
Mother's educational level (1-8 range) &     4.96  &   1.61  & 4.54 & 1. 64 & 4.81 & 1.64 \\
If owns a vehicle &    0.99   & 0.09   & 0.98 & 0.15 & 0.99 & 0.12 \\
\\
\textbf{Exclusion Restrictions}\\
\hline
Number of kids under 18 years old &     1.00  &  1.17  & 0.82 & 1.27 & 0.93 & 1.21 \\
If kids less than or equal to 15 years old  &    0.47  &  0.50  & 0.35 & 0.48 &  0.43 & 0.49  \\
Child care expenditure  &   1.47 &   3.91 & 0.25 & 1.64 & 1.04 & 3.35 \\
Husband's labor income &   61.49  &  81.16  & 62.76 & 195.65 & 61.93 & 132.85  \\
Household major expenditure  &   1.49 &   2.90   & 1.20 & 2.69 & 1.39 & 2.83 \\
\\
\textbf{Outcomes}\\
\hline
Labor income &      45,575.65&      43,701.59&      12,439.52&     15,746.09 &      41,909.04 &  42,822.50 \\
Log (Labor income)&       10.37&       0.98&       8.52&       1.58 &       10.17&       1.21 \\
If in the labor force  &       1     &        0    &      0      &     0       &       0.65 &   0.48  \\
\\
\textbf{Number of Observations (N)}&       \multicolumn{2}{c}{1,294}            &       \multicolumn{2}{c}{695}            &       \multicolumn{2}{c}{1,989}                      \\
\hline \hline
\end{longtable}
}
\end{center}

%{\footnotesize Notes: 161 women (out of 695 women not in the labor force) who report being neither employed nor (temporarily) unemployed also report having positive labor income. A set of dummy variables for the current state of the individuals and a set of indicators reflecting Beale-Ross rural-urban continuum code for the individual's current residence are omitted from the table with summary statistics but used as part of the explanatory variables in our analysis. }



\begin{center}
\begin{table}[th]
\caption{\textbf{Estimation results}} %First-Step Estimation Results Selected by Lasso}}
\label{tb:AllEstRes}
{\footnotesize 
\begin{tabular}{lcccccccccccc}
\hline
\hline


\multicolumn{8}{c}{\textbf{Earnings equation}}\\
\hline
\multicolumn{1}{c}{} &
  \multicolumn{1}{c}{(1)} &
  \multicolumn{1}{c}{(2)} &
  \multicolumn{1}{c}{(3)} &
  \multicolumn{1}{c}{(4)} &
  \multicolumn{1}{c}{(5)} &
  \multicolumn{1}{c}{(6)} &
  \multicolumn{1}{c}{(7)} \\
\cline{1-8}
\multicolumn{1}{l}{Education (years)} &
  \multicolumn{1}{c}{0.053} &
  \multicolumn{1}{c}{0.047} &
  \multicolumn{1}{c}{0.089***} &
  \multicolumn{1}{c}{0.056*} &
  \multicolumn{1}{c}{0.089***} &
  \multicolumn{1}{c}{0.060*} &
  \multicolumn{1}{c}{0.075***} \\
\multicolumn{1}{l}{} &
  \multicolumn{1}{c}{(0.034)} &
  \multicolumn{1}{c}{(0.040)} &
  \multicolumn{1}{c}{(0.023)} &
  \multicolumn{1}{c}{(0.034)} &
  \multicolumn{1}{c}{(0.023)} &
  \multicolumn{1}{c}{(0.033)} &
  \multicolumn{1}{c}{(0.016)} \\
\multicolumn{1}{l}{Experience} &
  \multicolumn{1}{c}{0.026***} &
  \multicolumn{1}{c}{0.006} &
  \multicolumn{1}{c}{} &
  \multicolumn{1}{c}{} &
  \multicolumn{1}{c}{-0.000} &
  \multicolumn{1}{c}{-0.018} &
  \multicolumn{1}{c}{0.001} \\
\multicolumn{1}{l}{} &
  \multicolumn{1}{c}{(0.007)} &
  \multicolumn{1}{c}{(0.018)} &
  \multicolumn{1}{c}{} &
  \multicolumn{1}{c}{} &
  \multicolumn{1}{c}{(0.015)} &
  \multicolumn{1}{c}{(0.021)} &
  \multicolumn{1}{c}{(0.014)} \\
\multicolumn{1}{l}{Experience$^2$} &
  \multicolumn{1}{c}{} &
  \multicolumn{1}{c}{0.001} &
  \multicolumn{1}{c}{} &
  \multicolumn{1}{c}{} &
  \multicolumn{1}{c}{} &
  \multicolumn{1}{c}{0.001} &
  \multicolumn{1}{c}{0.000} \\
\multicolumn{1}{l}{} &
  \multicolumn{1}{c}{} &
  \multicolumn{1}{c}{(0.001)} &
  \multicolumn{1}{c}{} &
  \multicolumn{1}{c}{} &
  \multicolumn{1}{c}{} &
  \multicolumn{1}{c}{(0.001)} &
  \multicolumn{1}{c}{(0.000)} \\
\multicolumn{1}{l}{Age} &
  \multicolumn{1}{c}{} &
  \multicolumn{1}{c}{} &
  \multicolumn{1}{c}{0.019***} &
  \multicolumn{1}{c}{-0.090} &
  \multicolumn{1}{c}{0.018***} &
  \multicolumn{1}{c}{-0.086} &
  \multicolumn{1}{c}{0.002} \\
\multicolumn{1}{l}{} &
  \multicolumn{1}{c}{} &
  \multicolumn{1}{c}{} &
  \multicolumn{1}{c}{(0.004)} &
  \multicolumn{1}{c}{(0.057)} &
  \multicolumn{1}{c}{(0.005)} &
  \multicolumn{1}{c}{(0.055)} &
  \multicolumn{1}{c}{(0.022)} \\
\multicolumn{1}{l}{Age$^2$} &
  \multicolumn{1}{c}{} &
  \multicolumn{1}{c}{} &
  \multicolumn{1}{c}{} &
  \multicolumn{1}{c}{0.001*} &
  \multicolumn{1}{c}{} &
  \multicolumn{1}{c}{0.001*} &
  \multicolumn{1}{c}{0.000} \\
\multicolumn{1}{l}{} &
  \multicolumn{1}{c}{} &
  \multicolumn{1}{c}{} &
  \multicolumn{1}{c}{} &
  \multicolumn{1}{c}{(0.001)} &
  \multicolumn{1}{c}{} &
  \multicolumn{1}{c}{(0.001)} &
  \multicolumn{1}{c}{(0.000)} \\
  \\
\multicolumn{1}{l}{$\lambda$} &
  \multicolumn{1}{c}{-1.336***} &
  \multicolumn{1}{c}{-1.463***} &
  \multicolumn{1}{c}{-0.838***} &
  \multicolumn{1}{c}{-1.584***} &
  \multicolumn{1}{c}{-0.924***} &
  \multicolumn{1}{c}{-1.581***} &
  \multicolumn{1}{c}{-0.942***} \\
\multicolumn{1}{l}{} &
  \multicolumn{1}{c}{(0.388)} &
  \multicolumn{1}{c}{(0.484)} &
  \multicolumn{1}{c}{(0.146)} &
  \multicolumn{1}{c}{(0.456)} &
  \multicolumn{1}{c}{(0.171)} &
  \multicolumn{1}{c}{(0.455)} &
  \multicolumn{1}{c}{(0.156)} \\
  \\
\multicolumn{1}{l}{{\bf Number of observations (N)}} &
  \multicolumn{1}{c}{1294} &
  \multicolumn{1}{c}{1294} &
  \multicolumn{1}{c}{1294} &
  \multicolumn{1}{c}{1294} &
  \multicolumn{1}{c}{1294} &
  \multicolumn{1}{c}{1294} &
  \multicolumn{1}{c}{1294} \\
\multicolumn{1}{l}{{\bf Number of controls}} &
  \multicolumn{1}{c}{35} &
  \multicolumn{1}{c}{34} &
  \multicolumn{1}{c}{34} &
  \multicolumn{1}{c}{34} &
  \multicolumn{1}{c}{35} &
  \multicolumn{1}{c}{34} &
  \multicolumn{1}{c}{} \\
\multicolumn{1}{l}{{\bf Number of selected controls}} &
  \multicolumn{1}{c}{19} &
  \multicolumn{1}{c}{19} &
  \multicolumn{1}{c}{20} &
  \multicolumn{1}{c}{21} &
  \multicolumn{1}{c}{21} &
  \multicolumn{1}{c}{21} &
  \multicolumn{1}{c}{} \\
\hline
\hline

\multicolumn{8}{c}{\textbf{Labor force participation equation}}\\
\hline
\multicolumn{1}{c}{} &
  \multicolumn{1}{c}{(1)} &
  \multicolumn{1}{c}{(2)} &
  \multicolumn{1}{c}{(3)} &
  \multicolumn{1}{c}{(4)} &
  \multicolumn{1}{c}{(5)} &
  \multicolumn{1}{c}{(6)} &
  \multicolumn{1}{c}{(7)} \\
\cline{1-8}
\multicolumn{1}{l}{Education (years)} &
  \multicolumn{1}{c}{0.066***} &
  \multicolumn{1}{c}{0.066***} &
  \multicolumn{1}{c}{0.074***} &
  \multicolumn{1}{c}{0.074***} &
  \multicolumn{1}{c}{0.070***} &
  \multicolumn{1}{c}{0.069***} &
  \multicolumn{1}{c}{0.067***} \\
\multicolumn{1}{l}{} &
  \multicolumn{1}{c}{(0.024)} &
  \multicolumn{1}{c}{(0.024)} &
  \multicolumn{1}{c}{(0.025)} &
  \multicolumn{1}{c}{(0.025)} &
  \multicolumn{1}{c}{(0.025)} &
  \multicolumn{1}{c}{(0.025)} &
  \multicolumn{1}{c}{(0.026)} \\
\multicolumn{1}{l}{Experience} &
  \multicolumn{1}{c}{0.012} &
  \multicolumn{1}{c}{0.012} &
  \multicolumn{1}{c}{} &
  \multicolumn{1}{c}{} &
  \multicolumn{1}{c}{0.042***} &
  \multicolumn{1}{c}{0.041***} &
  \multicolumn{1}{c}{0.042***} \\
\multicolumn{1}{l}{} &
  \multicolumn{1}{c}{(0.013)} &
  \multicolumn{1}{c}{(0.013)} &
  \multicolumn{1}{c}{} &
  \multicolumn{1}{c}{} &
  \multicolumn{1}{c}{(0.015)} &
  \multicolumn{1}{c}{(0.015)} &
  \multicolumn{1}{c}{(0.015)} \\
\multicolumn{1}{l}{Experience$^2$} &
  \multicolumn{1}{c}{-0.001***} &
  \multicolumn{1}{c}{-0.001***} &
  \multicolumn{1}{c}{} &
  \multicolumn{1}{c}{} &
  \multicolumn{1}{c}{} &
  \multicolumn{1}{c}{-0.001**} &
  \multicolumn{1}{c}{-0.001**} \\
\multicolumn{1}{l}{} &
  \multicolumn{1}{c}{(0.000)} &
  \multicolumn{1}{c}{(0.000)} &
  \multicolumn{1}{c}{} &
  \multicolumn{1}{c}{} &
  \multicolumn{1}{c}{} &
  \multicolumn{1}{c}{(0.000)} &
  \multicolumn{1}{c}{(0.000)} \\
\multicolumn{1}{l}{Age} &
  \multicolumn{1}{c}{} &
  \multicolumn{1}{c}{} &
  \multicolumn{1}{c}{0.168***} &
  \multicolumn{1}{c}{0.168***} &
  \multicolumn{1}{c}{0.160***} &
  \multicolumn{1}{c}{0.159***} &
  \multicolumn{1}{c}{0.164***} \\
\multicolumn{1}{l}{} &
  \multicolumn{1}{c}{} &
  \multicolumn{1}{c}{} &
  \multicolumn{1}{c}{(0.023)} &
  \multicolumn{1}{c}{(0.023)} &
  \multicolumn{1}{c}{(0.022)} &
  \multicolumn{1}{c}{(0.023)} &
  \multicolumn{1}{c}{(0.020)} \\
\multicolumn{1}{l}{Age$^2$} &
  \multicolumn{1}{c}{} &
  \multicolumn{1}{c}{} &
  \multicolumn{1}{c}{-0.002***} &
  \multicolumn{1}{c}{-0.002***} &
  \multicolumn{1}{c}{-0.002***} &
  \multicolumn{1}{c}{-0.002***} &
  \multicolumn{1}{c}{-0.002***} \\
\multicolumn{1}{l}{} &
  \multicolumn{1}{c}{} &
  \multicolumn{1}{c}{} &
  \multicolumn{1}{c}{(0.000)} &
  \multicolumn{1}{c}{(0.000)} &
  \multicolumn{1}{c}{(0.000)} &
  \multicolumn{1}{c}{(0.000)} &
  \multicolumn{1}{c}{(0.000)} \\
\multicolumn{1}{l}{Child care expenditure} &
  \multicolumn{1}{c}{0.073***} &
  \multicolumn{1}{c}{0.073***} &
  \multicolumn{1}{c}{0.075***} &
  \multicolumn{1}{c}{0.075***} &
  \multicolumn{1}{c}{0.072***} &
  \multicolumn{1}{c}{0.072***} &
  \multicolumn{1}{c}{0.085***} \\
\multicolumn{1}{l}{} &
  \multicolumn{1}{c}{(0.025)} &
  \multicolumn{1}{c}{(0.025)} &
  \multicolumn{1}{c}{(0.026)} &
  \multicolumn{1}{c}{(0.026)} &
  \multicolumn{1}{c}{(0.026)} &
  \multicolumn{1}{c}{(0.025)} &
  \multicolumn{1}{c}{(0.016)} \\
\multicolumn{1}{l}{If kids under 15} &
  \multicolumn{1}{c}{} &
  \multicolumn{1}{c}{} &
  \multicolumn{1}{c}{} &
  \multicolumn{1}{c}{} &
  \multicolumn{1}{c}{} &
  \multicolumn{1}{c}{} &
  \multicolumn{1}{c}{-0.288**} \\
\multicolumn{1}{l}{} &
  \multicolumn{1}{c}{} &
  \multicolumn{1}{c}{} &
  \multicolumn{1}{c}{} &
  \multicolumn{1}{c}{} &
  \multicolumn{1}{c}{} &
  \multicolumn{1}{c}{} &
  \multicolumn{1}{c}{(0.130)} \\
\multicolumn{1}{l}{Number of kids} &
  \multicolumn{1}{c}{} &
  \multicolumn{1}{c}{} &
  \multicolumn{1}{c}{} &
  \multicolumn{1}{c}{} &
  \multicolumn{1}{c}{} &
  \multicolumn{1}{c}{} &
  \multicolumn{1}{c}{-0.195***} \\
\multicolumn{1}{l}{} &
  \multicolumn{1}{c}{} &
  \multicolumn{1}{c}{} &
  \multicolumn{1}{c}{} &
  \multicolumn{1}{c}{} &
  \multicolumn{1}{c}{} &
  \multicolumn{1}{c}{} &
  \multicolumn{1}{c}{(0.047)} \\
\multicolumn{1}{l}{Husband's income} &
  \multicolumn{1}{c}{} &
  \multicolumn{1}{c}{} &
  \multicolumn{1}{c}{} &
  \multicolumn{1}{c}{} &
  \multicolumn{1}{c}{} &
  \multicolumn{1}{c}{} &
  \multicolumn{1}{c}{-0.001**} \\
\multicolumn{1}{l}{} &
  \multicolumn{1}{c}{} &
  \multicolumn{1}{c}{} &
  \multicolumn{1}{c}{} &
  \multicolumn{1}{c}{} &
  \multicolumn{1}{c}{} &
  \multicolumn{1}{c}{} &
  \multicolumn{1}{c}{(0.000)} \\
\multicolumn{1}{l}{Major expenses} &
  \multicolumn{1}{c}{} &
  \multicolumn{1}{c}{} &
  \multicolumn{1}{c}{} &
  \multicolumn{1}{c}{} &
  \multicolumn{1}{c}{} &
  \multicolumn{1}{c}{} &
  \multicolumn{1}{c}{0.014} \\
\multicolumn{1}{l}{} &
  \multicolumn{1}{c}{} &
  \multicolumn{1}{c}{} &
  \multicolumn{1}{c}{} &
  \multicolumn{1}{c}{} &
  \multicolumn{1}{c}{} &
  \multicolumn{1}{c}{} &
  \multicolumn{1}{c}{(0.012)} \\
\\
\multicolumn{1}{l}{{\bf Number of observations (N)}} &
  \multicolumn{1}{c}{1989} &
  \multicolumn{1}{c}{1989} &
  \multicolumn{1}{c}{1989} &
  \multicolumn{1}{c}{1989} &
  \multicolumn{1}{c}{1989} &
  \multicolumn{1}{c}{1989} &
  \multicolumn{1}{c}{1989} \\
\multicolumn{1}{l}{{\bf Number of controls}} &
  \multicolumn{1}{c}{35} &
  \multicolumn{1}{c}{34} &
  \multicolumn{1}{c}{34} &
  \multicolumn{1}{c}{34} &
  \multicolumn{1}{c}{35} &
  \multicolumn{1}{c}{34} &
  \multicolumn{1}{c}{} \\
\multicolumn{1}{l}{{\bf Number of selected controls}} &
  \multicolumn{1}{c}{21} &
  \multicolumn{1}{c}{21} &
  \multicolumn{1}{c}{22} &
  \multicolumn{1}{c}{22} &
  \multicolumn{1}{c}{23} &
  \multicolumn{1}{c}{22} &
  \multicolumn{1}{c}{} \\
\hline
\hline
\end{tabular}
}
\end{table}
\end{center}

% table for probit step
%\newpage
%\begin{center}
\begin{table}[th]
\caption{\textbf{Estimation results for labor force participation -- ASK DI ABOUT No of vars}} %First-Step Estimation Results Selected by Lasso}}
\label{tb:ProbitEstRes}
{\footnotesize 
\begin{tabular}{lcccccccccccc}
\hline
\hline
\multicolumn{1}{c}{} &
  \multicolumn{1}{c}{(1)} &
  \multicolumn{1}{c}{(2)} &
  \multicolumn{1}{c}{(3)} &
  \multicolumn{1}{c}{(4)} &
  \multicolumn{1}{c}{(5)} &
  \multicolumn{1}{c}{(6)} &
  \multicolumn{1}{c}{(7)} \\
\cline{1-8}
\multicolumn{1}{l}{Education (years)} &
  \multicolumn{1}{c}{0.066***} &
  \multicolumn{1}{c}{0.066***} &
  \multicolumn{1}{c}{0.074***} &
  \multicolumn{1}{c}{0.074***} &
  \multicolumn{1}{c}{0.070***} &
  \multicolumn{1}{c}{0.069***} &
  \multicolumn{1}{c}{0.067***} \\
\multicolumn{1}{l}{} &
  \multicolumn{1}{c}{(0.024)} &
  \multicolumn{1}{c}{(0.024)} &
  \multicolumn{1}{c}{(0.025)} &
  \multicolumn{1}{c}{(0.025)} &
  \multicolumn{1}{c}{(0.025)} &
  \multicolumn{1}{c}{(0.025)} &
  \multicolumn{1}{c}{(0.026)} \\
\multicolumn{1}{l}{Experience} &
  \multicolumn{1}{c}{0.012} &
  \multicolumn{1}{c}{0.012} &
  \multicolumn{1}{c}{} &
  \multicolumn{1}{c}{} &
  \multicolumn{1}{c}{0.042***} &
  \multicolumn{1}{c}{0.041***} &
  \multicolumn{1}{c}{0.042***} \\
\multicolumn{1}{l}{} &
  \multicolumn{1}{c}{(0.013)} &
  \multicolumn{1}{c}{(0.013)} &
  \multicolumn{1}{c}{} &
  \multicolumn{1}{c}{} &
  \multicolumn{1}{c}{(0.015)} &
  \multicolumn{1}{c}{(0.015)} &
  \multicolumn{1}{c}{(0.015)} \\
\multicolumn{1}{l}{Experience$^2$} &
  \multicolumn{1}{c}{-0.001***} &
  \multicolumn{1}{c}{-0.001***} &
  \multicolumn{1}{c}{} &
  \multicolumn{1}{c}{} &
  \multicolumn{1}{c}{} &
  \multicolumn{1}{c}{-0.001**} &
  \multicolumn{1}{c}{-0.001**} \\
\multicolumn{1}{l}{} &
  \multicolumn{1}{c}{(0.000)} &
  \multicolumn{1}{c}{(0.000)} &
  \multicolumn{1}{c}{} &
  \multicolumn{1}{c}{} &
  \multicolumn{1}{c}{} &
  \multicolumn{1}{c}{(0.000)} &
  \multicolumn{1}{c}{(0.000)} \\
\multicolumn{1}{l}{Age} &
  \multicolumn{1}{c}{} &
  \multicolumn{1}{c}{} &
  \multicolumn{1}{c}{0.168***} &
  \multicolumn{1}{c}{0.168***} &
  \multicolumn{1}{c}{0.160***} &
  \multicolumn{1}{c}{0.159***} &
  \multicolumn{1}{c}{0.164***} \\
\multicolumn{1}{l}{} &
  \multicolumn{1}{c}{} &
  \multicolumn{1}{c}{} &
  \multicolumn{1}{c}{(0.023)} &
  \multicolumn{1}{c}{(0.023)} &
  \multicolumn{1}{c}{(0.022)} &
  \multicolumn{1}{c}{(0.023)} &
  \multicolumn{1}{c}{(0.020)} \\
\multicolumn{1}{l}{Age$^2$} &
  \multicolumn{1}{c}{} &
  \multicolumn{1}{c}{} &
  \multicolumn{1}{c}{-0.002***} &
  \multicolumn{1}{c}{-0.002***} &
  \multicolumn{1}{c}{-0.002***} &
  \multicolumn{1}{c}{-0.002***} &
  \multicolumn{1}{c}{-0.002***} \\
\multicolumn{1}{l}{} &
  \multicolumn{1}{c}{} &
  \multicolumn{1}{c}{} &
  \multicolumn{1}{c}{(0.000)} &
  \multicolumn{1}{c}{(0.000)} &
  \multicolumn{1}{c}{(0.000)} &
  \multicolumn{1}{c}{(0.000)} &
  \multicolumn{1}{c}{(0.000)} \\
\multicolumn{1}{l}{Child care expenditure} &
  \multicolumn{1}{c}{0.073***} &
  \multicolumn{1}{c}{0.073***} &
  \multicolumn{1}{c}{0.075***} &
  \multicolumn{1}{c}{0.075***} &
  \multicolumn{1}{c}{0.072***} &
  \multicolumn{1}{c}{0.072***} &
  \multicolumn{1}{c}{0.085***} \\
\multicolumn{1}{l}{} &
  \multicolumn{1}{c}{(0.025)} &
  \multicolumn{1}{c}{(0.025)} &
  \multicolumn{1}{c}{(0.026)} &
  \multicolumn{1}{c}{(0.026)} &
  \multicolumn{1}{c}{(0.026)} &
  \multicolumn{1}{c}{(0.025)} &
  \multicolumn{1}{c}{(0.016)} \\
\multicolumn{1}{l}{If kids under 15} &
  \multicolumn{1}{c}{} &
  \multicolumn{1}{c}{} &
  \multicolumn{1}{c}{} &
  \multicolumn{1}{c}{} &
  \multicolumn{1}{c}{} &
  \multicolumn{1}{c}{} &
  \multicolumn{1}{c}{-0.288**} \\
\multicolumn{1}{l}{} &
  \multicolumn{1}{c}{} &
  \multicolumn{1}{c}{} &
  \multicolumn{1}{c}{} &
  \multicolumn{1}{c}{} &
  \multicolumn{1}{c}{} &
  \multicolumn{1}{c}{} &
  \multicolumn{1}{c}{(0.130)} \\
\multicolumn{1}{l}{Number of kids} &
  \multicolumn{1}{c}{} &
  \multicolumn{1}{c}{} &
  \multicolumn{1}{c}{} &
  \multicolumn{1}{c}{} &
  \multicolumn{1}{c}{} &
  \multicolumn{1}{c}{} &
  \multicolumn{1}{c}{-0.195***} \\
\multicolumn{1}{l}{} &
  \multicolumn{1}{c}{} &
  \multicolumn{1}{c}{} &
  \multicolumn{1}{c}{} &
  \multicolumn{1}{c}{} &
  \multicolumn{1}{c}{} &
  \multicolumn{1}{c}{} &
  \multicolumn{1}{c}{(0.047)} \\
\multicolumn{1}{l}{Husband's income} &
  \multicolumn{1}{c}{} &
  \multicolumn{1}{c}{} &
  \multicolumn{1}{c}{} &
  \multicolumn{1}{c}{} &
  \multicolumn{1}{c}{} &
  \multicolumn{1}{c}{} &
  \multicolumn{1}{c}{-0.001**} \\
\multicolumn{1}{l}{} &
  \multicolumn{1}{c}{} &
  \multicolumn{1}{c}{} &
  \multicolumn{1}{c}{} &
  \multicolumn{1}{c}{} &
  \multicolumn{1}{c}{} &
  \multicolumn{1}{c}{} &
  \multicolumn{1}{c}{(0.000)} \\
\multicolumn{1}{l}{Major expenses} &
  \multicolumn{1}{c}{} &
  \multicolumn{1}{c}{} &
  \multicolumn{1}{c}{} &
  \multicolumn{1}{c}{} &
  \multicolumn{1}{c}{} &
  \multicolumn{1}{c}{} &
  \multicolumn{1}{c}{0.014} \\
\multicolumn{1}{l}{} &
  \multicolumn{1}{c}{} &
  \multicolumn{1}{c}{} &
  \multicolumn{1}{c}{} &
  \multicolumn{1}{c}{} &
  \multicolumn{1}{c}{} &
  \multicolumn{1}{c}{} &
  \multicolumn{1}{c}{(0.012)} \\
\\
\multicolumn{1}{l}{{\bf Number of observations (N)}} &
  \multicolumn{1}{c}{1989} &
  \multicolumn{1}{c}{1989} &
  \multicolumn{1}{c}{1989} &
  \multicolumn{1}{c}{1989} &
  \multicolumn{1}{c}{1989} &
  \multicolumn{1}{c}{1989} &
  \multicolumn{1}{c}{1989} \\
\multicolumn{1}{l}{{\bf Number of variables}} &
  \multicolumn{1}{c}{94} &
  \multicolumn{1}{c}{93} &
  \multicolumn{1}{c}{93} &
  \multicolumn{1}{c}{93} &
  \multicolumn{1}{c}{95} &
  \multicolumn{1}{c}{95} &
  \multicolumn{1}{c}{} \\
\multicolumn{1}{l}{{\bf Number of selected controls}} &
  \multicolumn{1}{c}{59} &
  \multicolumn{1}{c}{59} &
  \multicolumn{1}{c}{59} &
  \multicolumn{1}{c}{59} &
  \multicolumn{1}{c}{60} &
  \multicolumn{1}{c}{61} &
  \multicolumn{1}{c}{} \\
\hline
\hline
\end{tabular}
}
\end{table}
\end{center}

\begin{comment}

{\scriptsize Notes: The dependent variable is the indicator for whether the
individual is in labor force or not.  Column (1) reports a subset of the
first-step estimation results when work experience (in a quadratic form) enters
the set of the first-step explanatory variables but only work experience in a
linear form is present in the labor income equation and thus required to be
among the ``selected" controls in the first step. Column (2) is identical to
column (1) except that work experience in a quadratic form is required to be
among the ``selected" controls in the probit step. Columns (3) and (4) are
similar to columns (1) and (2) except that they produce a subset of the
first-step estimation results when the individual's age (not work experience) in
a linear and quadratic forms, respectively, is a part of the set of the
first-step explanatory variables. Columns (5) and (6) follow the logic of the
previous two pairs of columns but report a subset of the first-step estimation
results when both age and work experience enter the set of the explanatory
variables in the first step of our analysis.  Finally, column (7) reports a
subset of the estimated coefficients from the first step probit used in the
traditional two-step Heckman sample selection procedure where the total number
of the explanatory variables from probit specification (6) is used as a set of
the explanatory variables. The estimated parameters and their standard errors
for a full set of state dummy variables as well as a full set of urban-rural
dummy variables are omitted to preserve space. The remaining set of the
explanatory variables in all the probit specifications additionally contains a
set of the control variables and potential exclusion restrictions, which  are
provided in Table \ref{tb:BaselineXs}.  \textbf{Check the last two rows of the
table.} }




%\newpage
%\begin{center}
\begin{table}
\caption{\textbf{Estimation results for earnings equation%Second-Step Estimation Results
}}\label{tb:MainEstRes}
{\footnotesize 
\begin{tabular}{llllllll}
\hline
\hline
\multicolumn{1}{c}{} &
  \multicolumn{1}{c}{(1)} &
  \multicolumn{1}{c}{(2)} &
  \multicolumn{1}{c}{(3)} &
  \multicolumn{1}{c}{(4)} &
  \multicolumn{1}{c}{(5)} &
  \multicolumn{1}{c}{(6)} &
  \multicolumn{1}{c}{(7)} \\
\cline{1-8}
\multicolumn{1}{l}{Education (years)} &
  \multicolumn{1}{c}{0.053} &
  \multicolumn{1}{c}{0.047} &
  \multicolumn{1}{c}{0.089***} &
  \multicolumn{1}{c}{0.056*} &
  \multicolumn{1}{c}{0.089***} &
  \multicolumn{1}{c}{0.060*} &
  \multicolumn{1}{c}{0.075***} \\
\multicolumn{1}{l}{} &
  \multicolumn{1}{c}{(0.034)} &
  \multicolumn{1}{c}{(0.040)} &
  \multicolumn{1}{c}{(0.023)} &
  \multicolumn{1}{c}{(0.034)} &
  \multicolumn{1}{c}{(0.023)} &
  \multicolumn{1}{c}{(0.033)} &
  \multicolumn{1}{c}{(0.016)} \\
\multicolumn{1}{l}{Experience} &
  \multicolumn{1}{c}{0.026***} &
  \multicolumn{1}{c}{0.006} &
  \multicolumn{1}{c}{} &
  \multicolumn{1}{c}{} &
  \multicolumn{1}{c}{-0.000} &
  \multicolumn{1}{c}{-0.018} &
  \multicolumn{1}{c}{0.001} \\
\multicolumn{1}{l}{} &
  \multicolumn{1}{c}{(0.007)} &
  \multicolumn{1}{c}{(0.018)} &
  \multicolumn{1}{c}{} &
  \multicolumn{1}{c}{} &
  \multicolumn{1}{c}{(0.015)} &
  \multicolumn{1}{c}{(0.021)} &
  \multicolumn{1}{c}{(0.014)} \\
\multicolumn{1}{l}{Experience$^2$} &
  \multicolumn{1}{c}{} &
  \multicolumn{1}{c}{0.001} &
  \multicolumn{1}{c}{} &
  \multicolumn{1}{c}{} &
  \multicolumn{1}{c}{} &
  \multicolumn{1}{c}{0.001} &
  \multicolumn{1}{c}{0.000} \\
\multicolumn{1}{l}{} &
  \multicolumn{1}{c}{} &
  \multicolumn{1}{c}{(0.001)} &
  \multicolumn{1}{c}{} &
  \multicolumn{1}{c}{} &
  \multicolumn{1}{c}{} &
  \multicolumn{1}{c}{(0.001)} &
  \multicolumn{1}{c}{(0.000)} \\
\multicolumn{1}{l}{Age} &
  \multicolumn{1}{c}{} &
  \multicolumn{1}{c}{} &
  \multicolumn{1}{c}{0.019***} &
  \multicolumn{1}{c}{-0.090} &
  \multicolumn{1}{c}{0.018***} &
  \multicolumn{1}{c}{-0.086} &
  \multicolumn{1}{c}{0.002} \\
\multicolumn{1}{l}{} &
  \multicolumn{1}{c}{} &
  \multicolumn{1}{c}{} &
  \multicolumn{1}{c}{(0.004)} &
  \multicolumn{1}{c}{(0.057)} &
  \multicolumn{1}{c}{(0.005)} &
  \multicolumn{1}{c}{(0.055)} &
  \multicolumn{1}{c}{(0.022)} \\
\multicolumn{1}{l}{Age$^2$} &
  \multicolumn{1}{c}{} &
  \multicolumn{1}{c}{} &
  \multicolumn{1}{c}{} &
  \multicolumn{1}{c}{0.001*} &
  \multicolumn{1}{c}{} &
  \multicolumn{1}{c}{0.001*} &
  \multicolumn{1}{c}{0.000} \\
\multicolumn{1}{l}{} &
  \multicolumn{1}{c}{} &
  \multicolumn{1}{c}{} &
  \multicolumn{1}{c}{} &
  \multicolumn{1}{c}{(0.001)} &
  \multicolumn{1}{c}{} &
  \multicolumn{1}{c}{(0.001)} &
  \multicolumn{1}{c}{(0.000)} \\
  \\
\multicolumn{1}{l}{$\lambda$} &
  \multicolumn{1}{c}{-1.336***} &
  \multicolumn{1}{c}{-1.463***} &
  \multicolumn{1}{c}{-0.838***} &
  \multicolumn{1}{c}{-1.584***} &
  \multicolumn{1}{c}{-0.924***} &
  \multicolumn{1}{c}{-1.581***} &
  \multicolumn{1}{c}{-0.942***} \\
  \\
\multicolumn{1}{l}{} &
  \multicolumn{1}{c}{(0.388)} &
  \multicolumn{1}{c}{(0.484)} &
  \multicolumn{1}{c}{(0.146)} &
  \multicolumn{1}{c}{(0.456)} &
  \multicolumn{1}{c}{(0.171)} &
  \multicolumn{1}{c}{(0.455)} &
  \multicolumn{1}{c}{(0.156)} \\
\multicolumn{1}{l}{{\bf Number of observations (N)}} &
  \multicolumn{1}{c}{1294} &
  \multicolumn{1}{c}{1294} &
  \multicolumn{1}{c}{1294} &
  \multicolumn{1}{c}{1294} &
  \multicolumn{1}{c}{1294} &
  \multicolumn{1}{c}{1294} &
  \multicolumn{1}{c}{1294} \\
\multicolumn{1}{l}{{\bf Number of variables}} &
  \multicolumn{1}{c}{57} &
  \multicolumn{1}{c}{58} &
  \multicolumn{1}{c}{57} &
  \multicolumn{1}{c}{58} &
  \multicolumn{1}{c}{58} &
  \multicolumn{1}{c}{60} &
  \multicolumn{1}{c}{60} \\
\hline
\hline
\end{tabular}
}
\end{table}
\end{center}




{\scriptsize Notes: The dependent variable is the log of total annual labor
income. Columns (1)-(6) are based on the probit specifications (1)-(6) from
Table \ref{tb:ProbitEstRes}, respectively. Column (7) reports the estimated
coefficients from the second step of the traditional two-step Heckman sample
selection procedure where the total number of the explanatory variables from
probit specification (6) in Table \ref{tb:ProbitEstRes} is used as a set of the
explanatory variables in the traditional first step. The estimated parameters
and their standard errors for a full set of state dummy variables as well as a
full set of urban-rural dummy variables are omitted from all the columns to
preserve space. The last row reports the total number of the explanatory
variables used in the second step (including the constant term and $\lambda$).}

\end{comment}





%------------------------------------------------------------------------------
%	Appendix
%------------------------------------------------------------------------------
\clearpage
\section*{Appendix}
\appendix
%------------------------------------------------------------------------------
%	vector version of double selection linear Lasso
%------------------------------------------------------------------------------
\section{Neyman orthogonal estimation for  vectors}\label{sec:dsvec}
%Algorithm \ref{algo:dsregress} is based on a extension of \cite{belloni:14}.
%Here is the justification.

Consider an extension of the partial linear framework of \cite{belloni:14}:
\begin{align*}
	Y &= D' \theta_0 + g_0(X) + U &E(U|D, X) = 0\\ 
	D_1 &= m_{0,1}(X) + V_1	& E(V_1 |X) = 0\\
	\vdots			\nonumber \\
	D_{k} &= m_{0,k}(X) + V_{k} & E(V_k |X) = 0
\end{align*}
where $D$ and  $\theta$ are $k \times 1$-vectors, rather then scalars. The functions $g_0(X)$ and $m_{0, j}(X)$ are unknown but we can use approximately sparse linear models to approximate these functions so
that the approximation errors is small enough. We know that if $\theta$ is a scalar, we can apply the double selection lasso to
achieve valid inference for $\theta$. Here, we will show that the same arguments can be applied when when $\theta$ is a low-dimensional
vector.

We follow \cite{cherno/etal:18} and %to justify that double-selection
%lasso methods can provide valid inference for $\theta$ when it is a vector.
%Here is the roadmap.
show that
\begin{enumerate}
\item the moment condition implied by the Robinson-style
``partialling-out'' is Neyman orthogonal.
\item  the application of Theorem 3.1 of \cite{cherno/etal:18} provides the asymptotic distribution (note that cross-fitting only permits to  relax sparsity
requirement, and the ``partialling-out'' approach  still provides valid
inference for $\theta$ even without cross-fitting).
\item the ``partialling-out' approach is equivalent to the double
selection lasso.
\end{enumerate}

We start with a general definition of Neyman orthogonal moments  \cite[see, e.g.,][Definition 2.1]{cherno/etal:18}. Suppose we wish to estimate a low-dimensional parameter $\theta_0$ using the moment
conditions
\begin{align}
  \E \left [ \psi(W; \theta_0, \eta_0) \right ] = 0,
\end{align}
where $W$ contains the random variables and  $\eta_0$ is a nuisance parameter vector, which can be high-dimensional. Suppose we
use machine learning techniques to estimate $\eta_0$. In essence,  Neyman orthogonality means that
small mistakes in the estimation of $\eta$ will not disturb the consistent
estimation of $\theta_0$.

Formally, Neyman orthogonality depends on the concept of pathwise (Gateux)
derivative. Let $D_r$ denote the Gateux derivative of the moment condition $\psi$ with respect to $\eta$ in direction $r$. Then, 
\begin{align}
  D_r [\eta - \eta_0] = \partial_r \left\{ 
    	\E \left [  \psi(W; \theta_0, \eta_0 + (\eta - \eta_0)r)  \right ]
  \right \}
\end{align}
for all $r \in [0, 1)$.  When $D_r$ is evaluated at $r=0$, we denote it as
  $D_0[\eta - \eta_0]$.
\begin{defi}
The moment condition $\E \left [ \psi(W; \theta_0, \eta_0) \right ] = 0$ is Neyman orthogonal if 
\begin{align*}
  D_0[\eta - \eta_0] = 0.
\end{align*}
\end{defi}


%\subsection{On the double Lasso when $\theta$ is a vector}

%\subsubsection{Step 1: Neyman orthogonal scores for ``partialing-out'' approach}
Now we show Neyman orthogonality of the ``partialing-out'' approach. 
For notational simplicity, we group the equations for $D_j \quad (j=1,
\ldots, k)$. 
\begin{align}
	Y &= D' \theta_0 + g_0(X) + U, & E(U |D, X) = 0, \label{eq:yeq} \\ 
	D &= m_{0}(X) + V, & E(V|X) = 0, \label{eq:deq}
\end{align}
where $D = (D_1, D_2, \ldots, D_{k})'$, 
$m_0(X) = (m_{0, 1}(X), m_{0, 2}(X), \ldots, m_{0, k}(X))'$, and $V = (V_1,
V_2, \ldots, V_{k})'$. 
Plugging  Eq.~\ref{eq:deq} into \ref{eq:yeq}, we obtain the reduced form for
$Y$:
\begin{align}
	Y &= l_0(X) + B, & E(B | X) = 0,
\end{align}
where $l_0(X) = m_0(X)\theta_0 + g_0(X)$ and $B = U + V'\theta_0$.
Now we can show that the following Robinson-style ``partialling-out'' moment
condition is Neyman orthogonal.

\begin{prop} \label{prop:psi}
The moment condition using the function \begin{align} 
\psi\left[ W; \theta, \eta \right] = (Y - l(X) - (D - m(X))'\theta)(D - m(X))
\label{eq:psi}
\end{align}
where $W = (Y, X, D)$ and $\eta = (l, m)$,  is Neyman orthogonal.
\end{prop}

\begin{proof}
First we show that 
 $\E\left[ \psi \left( W; \theta_0, \eta_0 \right) \right] =
0$:
\begin{align*}
\E\left[ \psi \left( W; \theta_0, \eta_0 \right) \right] &=
\E \left[ 
(Y - l_0(X) - (D - m_0(X))'\theta_0)(D - m_0(X))
\right] \\
&= \E\left[ (B - V'\theta_0)V \right]\\
& = \E\left[ (U + V'\theta_0 - V'\theta_0)V \right] \\
& = \E\left[ UV \right] \\
& = 0,
\end{align*}
where the last equality holds because $\E( U |D, X) = 0$

Next, we prove that $D_0\left[ \eta - \eta_0 \right] = 0$. First note that
\begin{align*}
 \E& \left [  \psi(W; \theta_0, \eta_0 + (\eta - \eta_0)r)  \right ] \\
 &= \E\left[ 
 (Y - l_0(X) - r (l(X) - l_0(X)) - (D - m_0(X)-r(m(X)-m_0(X)))'\theta)
 (D - m_0(X) - r(m(X) - m_0(X))
 \right]
\end{align*}
Thus, we can compute $D_r[\eta - \eta_0]$ as follows: 
\begin{align*}
D_r&[\eta - \eta_0] =  
- \E\left[
(Y - l_0(X) - r (l(X) - l_0(X)) - (D - m_0(X)-r(m(X)-m_0(X)))'\theta_0)
(m(X) - m_0(X))
\right]  \\
&-
\E\left[
 (l(X) - l_0(X)) - (m(X)-m_0(X))'\theta_0)
 (D - m_0(X) - r(m(X) - m_0(X))
\right]
\end{align*}
Now, set $r=0$ and evaluate $D_0[\eta - \eta_0]$:
\begin{align*}
D_0[\eta - \eta_0]
&= -\E \left[
\left( Y - l_0(X) - (D - m_0(X))'\theta_0 \right)
\left(
m(X) - m_0(X)
\right)
\right] \\
&- \E \left[
\left(
(l(X) - l_0(X)) - (m(X)-m_0(X))'\theta_0)
\right)
\left(
D - m_0(X)
\right)
\right] \\
& = -\E\left[ U (m(X) - m_0(X) \right]
- \E\left[ 
 (l(X) - l_0(X)) - (m(X)-m_0(X))'\theta_0)
 V
\right] \\
&= 0,
\end{align*}
where the last equality holds because of the law of iterated expectation.

Therefore, the moment condition is Neyman orthogonal.
\end{proof}

%\subsubsection{Step 2: asymptotic distribution of $\widehat{\theta}$}

Next, we provide the asymptotic distribution.  Let $\widehat{\theta}$ be a solution to $\frac{1}{n}\sum \psi_i(W; \theta, \eta) =0$,
given $\eta$. By Theorem 3.1 of \cite{cherno/etal:18},  $\widehat{\theta}$ is
a consistent estimator of $\theta_0$, and it is asymptotically normal with the rate of $\sqrt{N}$. Its variance is 
\begin{align}
\Sigma = J_0^{-1} \E (\psi(W; \theta_0, \eta_0) \psi(W; \theta_0, \eta_0)')
(J_0^{-1})', \label{eq:var}
\end{align}
where $J_0 = \E[ \partial \psi / \partial \theta |_{\theta = \theta_0}]$. We can simplify this expression to arrive at the following formula:
\begin{align}
\Sigma = (\E (VV') )^{-1} \E(VV'U^2) (\E(VV'))^{-1}.\label{eq:simple}
\end{align}

Eq.~(\ref{eq:simple}) leads to the following straightforward  estimator of $\Sigma$:
\begin{align}
\widehat{\Sigma} = 
\left(\frac{1}{n} \sum_i \widehat{V_i}\widehat{V_i}'\right)^{-1}
\left(
\frac{1}{n} \sum_i \widehat{V_i}\widehat{V_i}'\widehat{U_i}^2
\right)
\left(\frac{1}{n} \sum_i \widehat{V_i}\widehat{V_i}'\right)^{-1},
\label{eq:varhat}
\end{align}
where $\widehat{V_i}$ and $\widehat{U_i}$ are the residuals from Eqs.~(\ref{eq:yeq}) and (\ref{eq:deq}), respectively. 
Apparently, the variance estimator is the classic
heteroskedasticity-consistent estimator.

The last step is to show that double selection lasso is equivalent to the ``partialling-out''. We start with the equation 
\begin{align*}
	y = D'\theta_0 + g_0(X) + U.
\end{align*}
If we partial out $X$ from both $Y$ and $D$, the equation becomes
\begin{align}
y - l_0(X) = (D - m_0(X))'\theta_0 + U
\label{eq:ds}
\end{align}
It is easy to see that the solution for $\theta$ in Eq.~\ref{eq:ds} comes from the
Neyman orthogonal moment condition using the moment function in Eq.~\ref{eq:psi}. 

Therefore, double selection is equivalent to the Robinson-style
``partialling-out'' approach.

\section{Sketch of Proofs of Propositions \ref{prop:heckit} and \ref{prop:var} }\label{sec:proofs}

\textbf{Proposition \ref{prop:heckit}.} The proof is similar to that of Theorem 1 of \cite{bellonichernozhukovhansen2014} and essentially consists in verifying that the assumptions of that theorem are fulfilled given what we assumed in Assumptions 
 \ref{assump:heckman}-\ref{assump:r&n}. 

\textbf{Proof of Proposition \ref{prop:var}.} The construction of a variance estimator outlined in the paragraphs immediately preceding the proposition is similar to the standard correction for the first-step estimation error in the \citet{Heckman1979} procedure \cite[see, e.g.,][Section 16.10.2]{cameron/trivedi:05}. An important difference is that now we include the effect of the controls selected in the last step of the double selection lasso estimation when computing the residuals. 


%------------------------------------------------------------------------------
%\newpage
\section{{\tt dsheckman}: Stata command for DS-HECK estimator}
\label{sec:dsheckman_syntax}
%------------------------------------------------------------------------------

The syntax of {\tt dsheckman} is 

{\tt dsheckman} {\it depvar} {\it indepvars} [{\tt if}] [{\tt in}], 
	{\tt \underline{sel}ection({\it depvar\_s} = {\it indepvars\_s})}
	[selvars({\it varlist})]

\noindent
where 
\begin{itemize}
\item {\it depvar} specifies the dependent variable in the main equation, which
corresponds to $y_1$ in Eq.~(\ref{eq:y1}).

\item {\it indepvars} specifies the independent variables in the main equation,
which correspond to $\mathbf{x_1}$ in Eq.~(\ref{eq:y1}).

\item {\it depvar\_s} specifies the dependent variable in the selection
equation, which corresponds to $y_2$ in Eq.~(\ref{eq:y2}).

\item {\it indepvars\_s} specifies the independent variables in the selection
equation, which correspond to $\mathbf{x}$ and $\mathbf{z}$ in
Eq.~(\ref{eq:y2}).

\item {\tt selvars()} specifies $\mathbf{x}$ in Eq.~(\ref{eq:y2}). If this
option is not used, $\mathbf{x}$ is constructed in two steps: (a) run the
lasso probit of {\it devpar\_s} on {\it indepvars\_s} using the plugin penalty; denote the selected variables by $\mathbf{x_2}$; (b) construct $\mathbf{x}$ as the
union of $\mathbf{x_1}$ and $\mathbf{x_2}$.

\end{itemize}

%------------------------------------------------------------------------------
%	reference
%------------------------------------------------------------------------------
\newpage
\bibliographystyle{ecta}
\bibliography{dantzig}





\end{document}
