

Using PSID in 2013, we estimate the married women's wage in the presence of
high-dimensional controls and possible sample selection bias.  

The selection equation is about whether women choose to work. It has many 
potential controls, including the continuous variables and dummy variables.
The main Equation is about married women's wages. The variables of interests
are experience ({\tt expr}), experience squared ({\tt $exper^2$}), education
level in years ({\tt educ}), age ({\tt age}), age squared ({\tt $age^2$}), and
whether work for the government ({\tt govern}).

We estimate four models. 
\begin{itemize}
\item Model 1 has many controls in the selection-equation.  

\item Model 2 is basically the same as Model 1, but it includes some extra
controls in the main Equation. These controls are current living state ({\tt
state}), race ({\tt race}), and occupation categories ({\tt occ}). 

\item Model 3 is basically the same as Model 1, but we add race ({\tt race}) as
variable of interest in the main Equation.

\item Model 4 is basically the same as Model 3, but we add extra controls in the
main Equation. These controls are current living state and occupation
categories.
\end{itemize}


\subsection{Estimation results}
First, we define the potential controls in the selection equation.
\begin{stlog}
. local vars_sel exper exper2 educ_level  childcare_expen_2012            ///
>         i.if_kidsle15 num_kids wage_husband exp_appl i.wtr_enrolled     ///
>         i.wtr_grad_hs i.wtr_attend_college i.wtr_cert_educ              ///
>         i.wtr_educ_usa  i.father_educ_usa i.mother_educ_usa             /// 
>         i.rural_urban i.own_vehicle i.current_state
{\smallskip}

\end{stlog}

We estimate Model 1 using a double-selection-lasso Heckman. There are 92
controls and Lasso selects 24 of them.  The estimate for $\lambda$ is
significant at $5\%$ level.  It indicates sample selection bias.

\begin{stlog}
. dsheckman lnwage educ_level exper, selection(inlf = `vars_sel')         
{\smallskip}
step 1: lasso probit to select vars
step 2: dsprobit of y2 on selected zvars
{\smallskip}
Double selection probit               Number of obs               =      1,989
                                      Number of controls          =         89
                                      Number of selected controls =         10
\HLI{13}{\TOPT}\HLI{64}
             {\VBAR}               Robust
        inlf {\VBAR} Coefficient  std. err.      z    P>|z|     [95\% conf. interval]
\HLI{13}{\PLUS}\HLI{64}
  educ_level {\VBAR}    .071556   .0228702     3.13   0.002     .0267313    .1163807
      exper2 {\VBAR}  -.0011333   .0003511    -3.23   0.001    -.0018214   -.0004451
childca{\tytilde}2012 {\VBAR}   .0726106   .0258034     2.81   0.005     .0220368    .1231845
       exper {\VBAR}   .0156069   .0119473     1.31   0.191    -.0078093    .0390231
       _cons {\VBAR}  -.7051439   .2897608    -2.43   0.015    -1.273065   -.1372231
\HLI{13}{\BOTT}\HLI{64}
step 3: compute lambda
step 4: dsregress y1 on xvars, lambda with controls
{\smallskip}
Double-selection-lasso Heckman        Number of obs               =      1,989
                                             Selected             =      1,294
                                             Nonselected          =        695
                                      Number of variables         =         93
                                      Number of selected controls =          3
                                      Number of main variables    =          2
{\smallskip}
\HLI{13}{\TOPT}\HLI{64}
             {\VBAR} Coefficient  Std. err.      z    P>|z|     [95\% conf. interval]
\HLI{13}{\PLUS}\HLI{64}
  educ_level {\VBAR}   .0544304   .0382198     1.42   0.154    -.0204791    .1293399
       exper {\VBAR}   .0320553   .0079836     4.02   0.000     .0164076    .0477029
      lambda {\VBAR}   -1.93624   .4908842    -3.94   0.000    -2.898355   -.9741244
\HLI{13}{\BOTT}\HLI{64}
Note: in the main equation, there are 2 variables; in the selection equation,
      3 among 93 variables are used to predict inverse mills ratio.

\end{stlog}

In Model 2, we now include extra controls such as current living state, race,
and occupation in the main Equation. There are 173 controls in this model, and
Lasso selects 34 of them. Interestingly, the coefficient for {\tt govern}
becomes significant at \%10 levels in this model. The other coefficients do not
change much compared with Model 1.
\begin{stlog}
. dsheckman lnwage educ_level exper, selection(inlf = `vars_sel') ///
>         selvars(num_kids educ_level exper)
{\smallskip}
step 1: set {\sltt{varsofinterest}} in selection equation
step 2: dsprobit of y2 on selected zvars
{\smallskip}
Double selection probit               Number of obs               =      1,989
                                      Number of controls          =         90
                                      Number of selected controls =         12
\HLI{13}{\TOPT}\HLI{64}
             {\VBAR}               Robust
        inlf {\VBAR} Coefficient  std. err.      z    P>|z|     [95\% conf. interval]
\HLI{13}{\PLUS}\HLI{64}
    num_kids {\VBAR}  -.1008661   .0440309    -2.29   0.022    -.1871651   -.0145671
  educ_level {\VBAR}   .0719182   .0229172     3.14   0.002     .0270014     .116835
       exper {\VBAR}   .0172596    .011989     1.44   0.150    -.0062385    .0407576
       _cons {\VBAR}  -.7082983   .2901592    -2.44   0.015       -1.277   -.1395968
\HLI{13}{\BOTT}\HLI{64}
step 3: compute lambda
step 4: dsregress y1 on xvars, lambda with controls
{\smallskip}
Double-selection-lasso Heckman        Number of obs               =      1,989
                                             Selected             =      1,294
                                             Nonselected          =        695
                                      Number of variables         =         93
                                      Number of selected controls =          3
                                      Number of main variables    =          2
{\smallskip}
\HLI{13}{\TOPT}\HLI{64}
             {\VBAR} Coefficient  Std. err.      z    P>|z|     [95\% conf. interval]
\HLI{13}{\PLUS}\HLI{64}
  educ_level {\VBAR}   .0954287   .0356875     2.67   0.007     .0254825    .1653749
       exper {\VBAR}   .0155112   .0163068     0.95   0.341    -.0164495    .0474719
      lambda {\VBAR}  -.9183578   .7577044    -1.21   0.226    -2.403431    .5667155
\HLI{13}{\BOTT}\HLI{64}
Note: in the main equation, there are 2 variables; in the selection equation,
      3 among 93 variables are used to predict inverse mills ratio.

\end{stlog}

In Model 3, we add race as a variable of interest in the main Equation, and
everything else is the same as Model 1.
\begin{stlog}
. dsheckman lnwage educ exper c.exper\#c.exper age c.age\#c.age i.govern  i.race 
> ///
>         if married_head == 1, sel(inlf = `controls')                    
{\smallskip}
Double-selection Heckman              Number of obs               =      4,086
                                      Selected                    =      2,521
                                      Nonselected                 =      1,565
                                      Number of controls          =         92
                                      Number of selected controls =         38
{\smallskip}
\HLI{13}{\TOPT}\HLI{64}
             {\VBAR}      Coef.   Std. Err.      z    P>|z|     [95\% Conf. Interval]
\HLI{13}{\PLUS}\HLI{64}
        educ {\VBAR}   .1252089   .0111604    11.22   0.000     .1033349    .1470829
       exper {\VBAR}   .0410847   .0068593     5.99   0.000     .0276408    .0545286
             {\VBAR}
     c.exper\#{\VBAR}
     c.exper {\VBAR}  -.0005526   .0003633    -1.52   0.128    -.0012647    .0001594
             {\VBAR}
         age {\VBAR}   .0637184   .0266334     2.39   0.017     .0115179     .115919
             {\VBAR}
 c.age\#c.age {\VBAR}  -.0009851   .0003594    -2.74   0.006    -.0016896   -.0002806
             {\VBAR}
    1.govern {\VBAR}  -.0196135   .0253868    -0.77   0.440    -.0693707    .0301438
             {\VBAR}
        race {\VBAR}
Black, Af..  {\VBAR}  -.0177732   .0313904    -0.57   0.571    -.0792973    .0437508
American ..  {\VBAR}  -.9221571   .1928185    -4.78   0.000    -1.300074   -.5442398
      Asian  {\VBAR}  -.1843727   .3269772    -0.56   0.573    -.8252363    .4564909
Native Ha..  {\VBAR}          0  (omitted)
      Other  {\VBAR}   .2042594   .1074979     1.90   0.057    -.0064327    .4149515
             {\VBAR}
      lambda {\VBAR}  -.6469446   .3129544    -2.07   0.039    -1.260324   -.0335652
       _cons {\VBAR}   7.139789   .4548134    15.70   0.000     6.248371    8.031207
\HLI{13}{\BOTT}\HLI{64}
{\smallskip}
. est store model3
{\smallskip}

\end{stlog}

In Model 4, we add extra controls, such as the living state and occupation
category, in the main Equation. Everything else is the same as Model 3.
\begin{stlog}
. dsheckman lnwage educ exper c.exper\#c.exper age c.age\#c.age i.govern i.race /
> //
>         if married_head == 1, sel(inlf = `controls')                        /
> //
>         extra_controls(i.state i.occ)
{\smallskip}
Double-selection Heckman              Number of obs               =      4,086
                                      Selected                    =      2,521
                                      Nonselected                 =      1,565
                                      Number of controls          =        167
                                      Number of selected controls =         48
{\smallskip}
\HLI{13}{\TOPT}\HLI{64}
             {\VBAR}      Coef.   Std. Err.      z    P>|z|     [95\% Conf. Interval]
\HLI{13}{\PLUS}\HLI{64}
        educ {\VBAR}    .104469    .011236     9.30   0.000     .0824468    .1264912
       exper {\VBAR}   .0387878   .0065068     5.96   0.000     .0260347     .051541
             {\VBAR}
     c.exper\#{\VBAR}
     c.exper {\VBAR}  -.0005455   .0003465    -1.57   0.115    -.0012246    .0001336
             {\VBAR}
         age {\VBAR}   .0629125   .0258896     2.43   0.015     .0121698    .1136552
             {\VBAR}
 c.age\#c.age {\VBAR}  -.0009551   .0003487    -2.74   0.006    -.0016385   -.0002717
             {\VBAR}
    1.govern {\VBAR}   .0650142   .0267201     2.43   0.015     .0126438    .1173845
             {\VBAR}
        race {\VBAR}
Black, Af..  {\VBAR}  -.0199715   .0298898    -0.67   0.504    -.0785544    .0386114
American ..  {\VBAR}  -1.117056   .1837608    -6.08   0.000     -1.47722   -.7568912
      Asian  {\VBAR}   .0186225   .3095496     0.06   0.952    -.5880835    .6253286
Native Ha..  {\VBAR}          0  (omitted)
      Other  {\VBAR}   .0703524   .1011011     0.70   0.487    -.1278021     .268507
             {\VBAR}
      lambda {\VBAR}  -.6426614   .3067029    -2.10   0.036    -1.243788   -.0415348
       _cons {\VBAR}   7.506988   .4434182    16.93   0.000     6.637904    8.376071
\HLI{13}{\BOTT}\HLI{64}
{\smallskip}
. est store model4
{\smallskip}

\end{stlog}


Here are the estimation results of all models.
\begin{stlog}
. est table model1 model2 model3 model4, b star(0.1 0.05 0.01)
{\smallskip}
\HLI{13}{\TOPT}\HLI{64}
    Variable {\VBAR}    model1          model2          model3          model4      
\HLI{13}{\PLUS}\HLI{64}
        educ {\VBAR}  .12663029***    .10636322***     .1252089***    .10446901***  
       exper {\VBAR}  .04580829***    .04389429***     .0410847***    .03878783***  
             {\VBAR}
     c.exper\#{\VBAR}
     c.exper {\VBAR}   -.000725**    -.00071061**    -.00055265      -.00054552     
             {\VBAR}
         age {\VBAR}  .04810982*      .04398279*      .06371844**     .06291247**   
             {\VBAR}
 c.age\#c.age {\VBAR} -.00074352**     -.0006758**    -.00098508***   -.00095512***  
             {\VBAR}
      govern {\VBAR}
          1  {\VBAR} -.02642255       .05154112*     -.01961347       .06501418**   
             {\VBAR}
      lambda {\VBAR} -.61597198**    -.60992137**    -.64694459**    -.64266139**   
             {\VBAR}
        race {\VBAR}
Black, Af..  {\VBAR}                                 -.01777323      -.01997149     
American ..  {\VBAR}                                 -.92215709***   -1.1170558***  
      Asian  {\VBAR}                                 -.18437269       .01862255     
Native Ha..  {\VBAR}                                          0               0     
      Other  {\VBAR}                                  .20425941*      .07035244     
             {\VBAR}
       _cons {\VBAR}   7.267417***    7.6967934***    7.1397887***    7.5069877***  
\HLI{13}{\BOTT}\HLI{64}
                                           legend: * p<.1; ** p<.05; *** p<.01

\end{stlog}

Finally, if we try to estimate Model 1 using traditional Heckman, the probit for
the selection Equation does not even converge because of high-dimensional
controls. 
