
\section{User's guide to sj.sty}

The {\sl Stata Journal\/} is produced using \texttt{statapress.cls} and
\texttt{sj.sty}, a {\LaTeXe} document class and package, respectively, each
developed and maintained at StataCorp by the Stata Press staff.  These files
manage the look and feel of each article in the {\sl Stata Journal}.

\subsection{The title page}

Each insert must begin with title-generating commands.  For example,

\begin{stverbatim}
\begin{verbatim}
\inserttype[st0001]{article}
\author{short author list}{%
  First author\\First affiliation\\City, State/Country\\Email address
  \and
  Second author\\Second affiliation\\City, State/Country\\Email address
}
\title[short toc title]{Long title for first page of journal insert}
\maketitle
\end{verbatim}
\end{stverbatim}

Here \verb+\inserttype+ identifies the tag (for example, st0001) associated
with the journal insert and the insert type (for example, article).  The default
\verb+\inserttype+ is ``notag'', possibly with a number appended.
\verb+\author+ identifies the short and long versions of the list of
authors (that is, J. M. Doe for the short title and John Michael Doe for the 
long).  \verb+\title+ identifies the short (optional) and long (required)
versions of the title of the journal insert.  The optional argument to
\verb+\title+ is used as the even-numbered page header.  If the optional
argument to \verb+\title+ is not given, the long title is used.  The required
argument to \verb+\title+ is placed in the table of contents with the short
author list.  Titles should not have any font changes or {\TeX} macros in
them.  \verb+\maketitle+ must be the last command of this sequence; it uses
the information given in the previous commands to generate the title for a new
journal insert.

\clearpage
\subsection{The abstract}

The abstract is generated using the \texttt{abstract} environment.  The
\verb+\keywords+ are also appended to the abstract.  Here is an
example abstract with keywords:

\begin{stverbatim}
\begin{verbatim}
\begin{abstract}
This is an example article.  You should change the \input{} line in
\texttt{main.tex} to point to your file.  If this is your first submission to
the {\sl Stata Journal}, please read the following ``getting started''
information.

\keywords{\inserttag, command name(s), keyword(s)}
\end{abstract}
\end{verbatim}
\end{stverbatim}

\verb+\inserttag+ will be replaced automatically with the tag
given in \verb+\inserttype+ (here st0001).

\subsection{Sectioning}

All sections are generated using the standard {\LaTeX} sectioning commands:\\
\verb+\section+, \verb+\subsection+, \dots.

Sections in articles are numbered.  If the optional short section title is
given, it will be put into bookmarks for the electronic version of the
journal; otherwise, the long section title is used.  Like article titles,
section titles should not have any font changes or {\TeX} macros in them.

\subsection{The bib option}

\textsc{Bib}{\TeX} is a program that formats citations and references
according to a bibliographic style.  The following two commands load the
bibliographic style file for the {\sl Stata Journal\/} (\texttt{sj.bst}) and
open the database of bibliographic entries (\texttt{sj.bib}):

\begin{stverbatim}
\begin{verbatim}
\bibliographystyle{sj}
\bibliography{sj}
\end{verbatim}
\end{stverbatim}

Here are some example citations:
%
\citet{akaike}, \citet*{benAkivaLerman}, \citet{dykePatterson},
\citet{greene03},
\citet*{kendallstuart}, \citet{hilbe93a}, \citet{hilbe94}, \citet{hilbe93b},
\citet{maddala83}, and \citet*{latexcompanion}.
%
They are generated by using
the \verb+\citet+ and \verb+\citet*+ commands from the \texttt{natbib}
package.  Here we test \verb+\citeb+ and \verb+\citebetal+:
%
\citeb{akaike}, \citeb{benAkivaLerman}, \citeb{dykePatterson},
\citeb{greene03}, \citeb{kendallstuart}, \citeb{hilbe93a}, \citeb{hilbe94},
\citeb{hilbe93b}, \citeb{maddala83}, and \citeb{latexcompanion}.
Sometimes using the \verb+\cite+ macros will result in an overfull line as
shown above.  The solution is to list the author names and the citation year
separately, for example,
\verb+Ben-Akiva and Lerman [\citeyear{benAkivaLerman}]+.

\clearpage
The \texttt{bib} option of \texttt{statapress.sty} indicates that citations
and references will be formatted using \textsc{Bib}{\TeX} and the
\texttt{natbib} package.  This option is the default (meaning that it need not
be supplied), but there is no harm in supplying it to the \texttt{statapress}
document class in the main {\LaTeX} driver file (for example,
\texttt{main.tex}).

\begin{stverbatim}
\begin{verbatim}
\documentclass[bib]{sj}
\end{verbatim}
\end{stverbatim}

\noindent
If you choose not to use \textsc{Bib}{\TeX}, you can use the \texttt{nobib}
option of \texttt{statapress.sty}.

\begin{stverbatim}
\begin{verbatim}
\documentclass[nobib]{statapress}
\end{verbatim}
\end{stverbatim}

\noindent
\textsc{Bib}{\TeX}{} and bibliographic styles are described in
\citet*{latexcompanion}.

\subsection{Author information}

The {\sl About the authors\/} section is generated by using the
\texttt{aboutauthors} environment.  There is also an \texttt{aboutauthor}
environment for journal inserts by one author.  For example,

\begin{stverbatim}
\begin{verbatim}
\begin{aboutauthor}
Text giving background about the author goes in here.

\end{aboutauthor}
\end{verbatim}
\end{stverbatim}

\endinput
